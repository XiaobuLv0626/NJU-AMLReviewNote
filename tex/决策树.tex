\chapter{决策树}

\section{决策树的基本流程}\label{sec:6.1}
决策树的基本策略是:\textbf{“分而治之”}。

决策树从根节点开始,不断地寻找“划分”属性,将数据集递归地不断划分为小集合。可以说,决策树是一个从根到叶的递归过程,每个中间节点对应一个属性测试,该递归的终止条件为:

\begin{itemize}
    \item 当前节点包含的样本已经全部属于同一类别,\textbf{无需划分};
    \item 当前\textbf{属性集为空},或所有样本在所有属性上取值相同,\textbf{无法划分};这种情况下,生成一个叶节点,类别为该节点所含样本最多的类别。
    \item 当前节点的\textbf{样本集为空,不能划分}。这种情况下,生成一个叶节点,类别为父节点所含样本最多的类别。
\end{itemize}
\section{划分选择}\label{sec:6.2}
决策树学习的关键在于\textbf{选择最优的划分标准}。我们希望决策树的分支节点所包含的样本尽可能属于同一类别,即\textbf{“纯度”}越高越好。

\subsection{信息增益}
信息增益的划分选择依赖于信息熵,信息熵(information entropy) 是度量样本集合纯度最常用的一种指标。假定当前样本集合$D$中第$k$类样本所占的比例为$p_k (k = 1,2,. . . , |\mathcal{Y}|)$,则D的信息熵定义为:
\[
\operatorname{Ent}(D)=-\sum_{k=1}^{\mid \mathcal{Y |}} p_{k} \log _{2} p_{k}
\]
$\operatorname{Ent}(D)$值越小,$D$的纯度越高。

假设某个属性$a$有V个可能的取值$\{a^1,a^2,...,a^V\}$,在决策树上就会产生V个分支节点,第v个节点包括了所有在a上取值为$a^v$的样本,记作$D^v$,我们可以计算出节点的信息熵,并且为其乘上一个系数,该系数由该取值对应的样本数决定,被称之为“信息增益”。
\[
\operatorname{Gain}(D, a)=\operatorname{Ent}(D)-\sum_{v=1}^{V} \frac{\left|D^{v}\right|}{|D|} \operatorname{Ent}\left(D^{v}\right)
\]
\marginpar{\footnotesize 信息增益偏好取值数多的属性。}
信息增益越大,用属性a进行划分得到的纯度提升越大。

\subsection{增益率}
增益率为平衡属性偏好的不利影响,引入“增益率”选择最优划分属性。
\marginpar{\footnotesize 增益率偏好取值数较少的属性。}
\[
\operatorname{Gain\_ ratio}(D, a)=\frac{\operatorname{Gain}(D, a)}{\operatorname{IV}(a)} = \frac{\operatorname{Gain}(D,a)}{-\frac{\left|D^{v}\right|}{|D|} \log _{2} \frac{\left|D^{v}\right|}{|D|}}
\]

\section{剪枝处理}\label{sec:6.3}
决策树的决策分支过多,可能导致将训练集自身的一些特性当作所有数据均具有的一般性质,从而导致过拟合。因此,需要对决策树进行剪枝从而提升泛化性。

测试泛化性能是否提升的方法:留出法。(留出一部分数据进行验证)

剪枝处理有两种测试思路:
\begin{itemize}
    \item 预剪枝:预剪枝的思路是\textbf{边建树边剪枝},在划分前先估计该节点的划分是否能提升泛化性能(计算验证集精度),若不能提升精度则不允许划分。

    其优点在于能够降低过拟合风险,减少训练与测试时间开销,但有可能导致欠拟合。

    \item 后剪枝:后剪枝的思路是\textbf{先建树后剪枝},考察建树之后每个节点替换为叶节点是否能提升验证集精度,可以则剪枝缩点。

    其优点在于能够降低欠拟合风险,泛化性能更强,但其训练时间开销大。
\end{itemize}

\section{多变量决策树}\label{sec:6.4}

单变量决策树的分类边界是与坐标轴平行的,而多变量决策树则是对属性的线性组合,每个节点都是一个形如$\sum_{i=1}^d w_i a_i=t$的线性分类器,其中$ w_i$是属性$a_i$的权值,$w_i$与$t$可在该节点所含的样本集与属性集上学得。

\section{本章往年考试题目}\label{sec:6.5}

\ex{2022年考试原题}{ex_ref}{决策树为何容易过拟合?解决方案是什么?}

详见\ref{sec:6.3}节。

\ex{2023年考试原题}{ex_ref}{决策树最优划分的两个准则,以及他们在属性选择上的偏好是什么?}

详见\ref{sec:6.2}节。