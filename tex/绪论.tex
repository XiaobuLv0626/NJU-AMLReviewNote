\chapter{绪论}
\section{机器学习是什么?}\label{sec:1.1}
\begin{Keynote}
机器利用\textbf{数据}学习人类的\textbf{经验},从而不断提升系统自身\textbf{性能}的过程。
\end{Keynote}
一个典型的机器学习过程:

有类别标记的训练数据 -> 学习算法训练 -> 模型 <- 类别未知的新数据样本

\section{机器学习的历史与未来?}\label{sec:1.2}

机器学习的历史:
\begin{itemize}
 \item 推理期(60年代——80年代初期)
 \item 知识期(80年代初期——90年代中期)
 \item 学习期(90年代中期——今)
\end{itemize}
机器学习的未来:\textbf{可信稳健协同机器学习}
\section{机器学习与其他领域的关系?}\label{sec:1.3}
与数据挖掘领域:ML是数据挖掘的关键支撑技术,但ML更偏技术,DM更偏应用。

与数据科学领域:ML是数据科学实现智能化的关键步骤,没有ML分析数据,大数据无法利用。

与计算机视觉/自然语言处理领域:ML是CV/NLP的核心技术。

与神经科学领域:ML的发展过程中经常受到神经科学的启发。
\section{机器学习前沿进展}\label{sec:1.4}
学术期刊:
\begin{itemize}
\item AIJ 《Artificial Intelligence》
\item JMLR 《Journal of Machine Learning Research》
\item TPAMI《IEEE Trans. on Pattern Analysis and Machine Intelligence》
\item TKDE《IEEE Trans. on Knowledge and Data Engineering》
\item MLJ 《Machine Learning》
\item TNNLS 《IEEE Trans. on Neural Network and Learning Systems
\end{itemize}
学术会议:
\begin{itemize}
\item ICML (International Conference on Machine Learning)
\item NeurIPS (Neural Information Processing Systems)
\item KDD (ACMSIGKDD Conf. on Knowledge Discovery and Data Mining)
\item AAAI (AAAI conference on Artificial Intelligence)
\item IJCAI (International Joint Conference on Artificial Intelligence)
\item ICLR (International Conference on Learning Representation)
\item MLA (中国机器学习及其应用研讨会)
\end{itemize}

\section{本章往年考试题目}\label{sec:1.5}
\ex{2021年考试原题}{ex_ref}{简述机器学习的鲁棒性?}
鲁棒性是机器学习性能的关键。
\begin{itemize}
 \item 对人类用户错误鲁棒
 \item 对网络攻击鲁棒
 \item 对错误目标鲁棒
 \item 对不正确模型鲁棒
 \item 对未建模现象鲁棒
\end{itemize}

\ex{2022/2023/2025年考试原题}{ex_ref}{什么是机器学习?举出两个应用实例。机器学习的未来挑战和发展?}
\textbf{定义}:机器利用\textbf{数据}学习人类的\textbf{经验},从而不断提升系统自身\textbf{性能}的过程。

应用实例略,未来发展略。

\textbf{挑战}:稳健性不足。
