\chapter*{写在前面} 

本笔记是南京大学研究生必修课程《高级机器学习》的复习笔记。

实际上,研究生阶段的同学可能并不会亲自去上《高级机器学习》的课程,而到期末,鸿篇巨制的西瓜书,PPT与公式推导也容易让同学们晕头转向,找不到合适的复习资料,更何况在南大的课程安排中,本课程属于闭卷考试,抱着一大摞PPT和近300页的教材阅读,实在是过于费时费力。

基于如上的思考,笔者决定对现有的高级机器学习资料进行整合,提出一份知识点整理笔记。这份笔记会包括:
\begin{itemize}
    \item 教材与PPT上涉及到的知识点详述;
    \item 一些关于重要公式推导的补充内容;
    \item 笔者所收集到的往年卷考试原题与相关解析。
\end{itemize}

笔者收集到的往年卷为2018-2023六年的考题,其中2021年的考题由于回忆不全,只包含一部分题目。根据2025年春季学期的考纲,高级机器学习的期末考试为10-11道题,每题10分,包括1道附加题,满分100,每章大概会出1道题。

这份复习笔记的诞生离不开南大学长学姐们公开资料的支援,在此感谢:
\begin{itemize}
    \item Maxwell Lyu 吕导的AML复习提纲(22年)(\url{https://blog.xlworkshop.ltd/post/nju-aml-2022/})
    \item Peanuts 学长/学姐的2022/2023年AML原题与解析(\url{https://zhuanlan.zhihu.com/p/702835146})
\end{itemize}

值得一提的是,囿于笔者水平以及写作时间不足,这份笔记不可避免地会出现考点的疏漏,病句错字,格式混乱等各种问题,还请各位多多包涵。

复习笔记基于Latex编写,在Github上开源,因此,希望贡献新的考试题目或者更正笔者出现的错误的各位,欢迎提交Pull Request,为这份笔记继续丰富内容。

只愿这份笔记能够帮助到所有南大CS研究生同学们,大家一起认真考试,投身科研。

\begin{flushright}
	2025.06.05

    清野千秋

    于常州楼作
\end{flushright}