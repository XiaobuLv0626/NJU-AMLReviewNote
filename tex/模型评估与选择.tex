\chapter{模型评估与选择}

\section{经验误差与过拟合}\label{sec:2.1}

\begin{itemize}
    \item 泛化误差:在“未来样本”(unseen instance)上的误差。
    \item 经验误差:在训练集合上的误差,也被称之为“训练误差”。
    \item 过拟合:把特性当成共性,特征学习过度。
    \item 欠拟合:把共性当成特性,特征学习不足。
\end{itemize}

\begin{figure}[!htbp]
	\centering
		\sidecaption{一个实际过拟合/欠拟合的例子\label{fig:example1}}{\includegraphics[width=.7\textwidth]{images/over-and-under.png}}
\end{figure}
%--------------------------------%
%--------------------------------%
\section{评估方法}\label{sec:2.2}
\begin{itemize}
    \item 留出法:将数据集划分为两个互斥的集合——训练与测试集,并尽可能保持数据分布以及类别比例的一致性;
    \item 交叉验证法:将数据集分层采样划分为K个大小相似的互斥子集,每次用k-1个子集的并集作为训练集,余下的子集作为测试集,最终返回测试结果的均值。
    \item 留一法:k=数据集的样本数。
\end{itemize}

上述方法从上到下变得越精确,但复杂度也更高,计算开销更大。

%--------------------------------%
%--------------------------------%
\section{性能度量}\label{sec:2.3}
性能度量是衡量模型泛化能力的评价标准,反映了任务需求;使用不同的性能度量往往会导致不同的评判结果。

\prp{性能度量}{pro_ref}{对于\textbf{回归任务},最常用的是均方误差MSE:
\[
E(f ; D)=\frac{1}{m} \sum_{i=1}^{m}\left(f\left(\boldsymbol{x}_{i}\right)-y_{i}\right)^{2}
\]

对于\textbf{分类任务},错误率与精度是最常用的性能度量:
\begin{enumerate}
    \item 分类错误率:$ E(f ; D)=\frac{1}{m} \sum_{i=1}^{m} \mathbb{I}\left(f\left(\boldsymbol{x}_{i}\right) \neq y_{i}\right) $
    \item 精度Accuracy:$\operatorname{acc}(f ; D) =\frac{1}{m} \sum_{i=1}^{m} \mathbb{I}\left(f\left(\boldsymbol{x}_{i}\right)=y_{i}\right) = 1-E(f ; D) . $
\end{enumerate}

除此之外,基于分类结果混淆矩阵,还有: 
\begin{itemize}
    \item 查准率:$ P = \frac{TP}{TP+FP} $
    \item 查全率:$ R = \frac{TP}{TP+FN} $
    \item F1度量(F1-Score):$ F_1 = \frac{2\times P\times R}{P+R} = \frac{2\times TP}{样例总数+TP-TN} $
    \item F-\beta 度量:$ F_\beta = \frac{(1+\beta^2)\times P\times R}{(\beta^2\times P)+R} $
\end{itemize}

以及AUC(Area Under Curve),基于真比例率(y)与假正例率(x)形成的ROC曲线,曲线下面积大小即为AUC:
\[
AUC = \frac{1}{2}\sum_{i=1}^{m-1}(x_{i+1}-x_i)\cdot (y_i+ y_{i+1})
\]
}

\begin{figure}[!htbp]
	\centering
		\sidecaption{分类结果混淆矩阵,上述公式中对应的值参见此处。\label{fig:example1}}{\includegraphics[width=.6\textwidth]{images/types.png}}
\end{figure}

%--------------------------------%
%--------------------------------%
\section{比较检验}\label{sec:2.4}
测试性能不等于泛化性能,且会随着测试集的变化而变化,机器学习算法本身有一定的随机性。因此,需要利用\textbf{假设检验}为学习器的性能比较提供数学理论依据。

\thm{二项检验}{thm_ref}{考虑假设“$\epsilon \le \epsilon_0$”, 则在$1 - \alpha$的概率内所能观测到的最大错误率如下式计算:
\[
\bar{\epsilon} = \min \epsilon \ \text{s.t.} \ \sum_{i=\epsilon\times m+1}^m\binom{m}{i}\epsilon_0^i(1-\epsilon_0)^{m-i} \lt \alpha
\]

此时,若测试错误率$\hat{\epsilon}$小于临界值$\bar{\epsilon}$,则我们称,在\alpha 的显著度下,假设不能被拒绝,泛化器的错误率不大于$\epsilon_0$。否则,该假设可被拒绝,即在\alpha 的显著度下可认为学习器的泛化错误率大于$\epsilon_0$.
}

\thm{T-检验}{thm_ref}{很多时候,我们并非只做一次留出法统计,而是通过多次留出法或是使用交叉验证法进行多次训练测试,得到k个测试错误率$\hat{\epsilon}_{1}, \hat{\epsilon}_{2}, \ldots, \hat{\epsilon}_{k}$,则平均错误率$\mu$与方差$\sigma^2$为:
\[
\begin{array}{c}\mu=\frac{1}{k} \sum_{i=1}^{k} \hat{\epsilon}_{i} \\ \sigma^{2}=\frac{1}{k-1} \sum_{i=1}^{k}\left(\hat{\epsilon}_{i}-\mu\right)^{2}\end{array}
\]
这k个测试错误率可看作泛化错误率$\epsilon_0$的独立采样,则变量
\[
\tau_{t}=\frac{\sqrt{k}\left(\mu-\epsilon_{0}\right)}{\sigma}
\]
服从自由度为$k - 1$的t分布。
因此,假设$\epsilon = \epsilon_0$对于显著度\alpha, 若$ [t_{-\alpha/2}, t_{\alpha/2}] $ 位于临界范围$ | \mu - \epsilon_0 | $内,则假设不能被拒绝,即可认为泛化错误率$\epsilon = \epsilon_0$,其置信度为$1 - \alpha$。
}

对两个学习器A和B,若我们使用k折交叉验证法得到的测试错误率分
别为 $ \epsilon_{1}^{A}, \ldots, \epsilon_{k}^{A} $ 和 $ \epsilon_{1}^{B}, \ldots, \epsilon_{k}^{B} $,其中$ \epsilon_i^A$和$\epsilon_i^B$是在相同的第i折训练/测试集上得到的结果,则可用k折交叉验证"成对t检验"来进行比较检验。其基本思想是:若两个学习器的性能相同,则\textbf{它们使用相同的训练/测试集得到的测试错误率}应相同,即$\epsilon_i^A=\epsilon_i^B$。


%--------------------------------%
%--------------------------------%
\section{偏差与方差}\label{sec:2.5}
对于回归任务,泛化误差可通过“偏差——方差分解”拆解为:
\begin{figure}[!htbp]
	\centering
	{\includegraphics[width=\textwidth]{images/bias-var.png}}
\end{figure}

\begin{Keynote}
泛化性能是由\textbf{学习算法的能力},\textbf{数据的充分性}以及\textbf{学习任务本身的难度}共同决定。
\end{Keynote}

\section{本章往年考试题目}\label{sec:2.6}
\ex{2023/2025年考试原题}{ex_ref}{从训练误差和泛化误差的角度阐述过拟合和欠拟合的含义,并举出1-2个机器学习模型例子说明如何克服过拟合和欠拟合。}
\begin{itemize}
    \item 过拟合:训练误差小,泛化误差大
    \item 欠拟合:训练误差和泛化误差都大
\end{itemize}

克服过拟合的方法:
\begin{itemize}
    \item 模型简单化
    \item 增加训练数据
    \item 增加正则化约束
    \item 提前结束训练
\end{itemize}

克服欠拟合的方法:
\begin{itemize}
    \item 模型复杂化
    \item 加入新的特征
    \item 降低正则化约束
\end{itemize}

\ex{2022年考试原题}{ex_ref}{理想模型为什么难以实现?评估方法,性能度量,比较检验分别解决什么问题?}
理想模型难以实现的原因:数据的可获得性和质量问题;模型的复杂性和计算资源的问题;泛化能力的局限性;人工智能的局限性;建模复杂性和不确定性。
\begin{itemize}
    \item 评估方法:解决的是“如何获取测试结果”;
    \item 性能度量:解决的是“如何评估性能优劣”;
    \item 比较检验:解决的是“如何判断实质差别”。
\end{itemize}
