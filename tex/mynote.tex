\documentclass[lang=en,12pt,twoside]{textbook}
% ----------------------------- Math Font ------------------------------------- %
\usepackage[T1]{fontenc}
\usepackage{newtx,anyfontsize}
\DeclareSymbolFont{CMlargesymbols}{OMX}{cmex}{m}{n}
\let\sum\relax
\let\prod\relax
\DeclareMathSymbol{\sum}{\mathop}{CMlargesymbols}{"50}
\DeclareMathSymbol{\prod}{\mathop}{CMlargesymbols}{"51}
%% \infty
\DeclareSymbolFont{symbolsCM}{OMS}{cmsy}{m}{n}
\SetSymbolFont{symbolsCM}{bold}{OMS}{cmsy}{b}{n}
\let\txinfty\infty
\DeclareMathSymbol{\infty}{\mathord}{symbolsCM}{"31}
\definecolor{nuanbai}{HTML}{f5f5f5}
\pagecolor{lightgray!6}
% ---------------------------------------------------------------------------- %
% \overfullrule=5pt
% \showboxdepth=\maxdimen
% \showboxbreadth=\maxdimen
\tracingonline=1
\tracingoutput=1

\begin{document}
%----------------------------------------------------------------------------------------
%	标题页信息
%----------------------------------------------------------------------------------------
\title{Machine Learning and Basic Mathematics}
\subtitle{\textit{Notes organizations}}
\author{Yi Ning}
\date{\today}
\publishers{}
%----------------------------------------------------------------------------------------

%----------------------------------------------------------------------------------------
%	插入自定义标题页
%----------------------------------------------------------------------------------------
\begin{titlepage} % 创建一个新的页面
%用来将图片从左下角开始平铺整个封面
	\AddToShipoutPicture*{%
	\AtPageLowerLeft{%
		\adjustbox{width=1.1\paperwidth, height=1.5\paperheight, keepaspectratio}{% 强制图片至纸张尺寸,但可能裁切
			\includegraphics{images/pexels-photo-3378916.jpeg}
		}
	}
}
\begin{flushleft} % 左对齐环境
	\setlength{\leftskip}{1cm} % 左侧缩进
	\makeatletter % 允许访问带有@字符的内部命令
	% \vspace*{4cm} % 标题距离页面顶端的空白
	% {\color{white}\Huge \textbf{\@title} \par} % 使用前文定义的title作为标题
	% \vspace{1cm} % 标题和子标题的间距
	% {\color{white}\Large \@subtitle \par} % 使用前文定义的subtitle作为子标题
	% \vfill % 作者信息前的垂直填充
	% {\color{white}\large \@author \par} % 作者名
	% {\color{white}\large \@publishers \par} % 出版者
	% {\color{white}\large \today\par} % 日期,默认为当天
        \begin{tikzpicture}[overlay,remember picture]
        \begin{pgfonlayer}{bottom}
            \fill[dblue!10,opacity=0.1] (current page.south west) rectangle ++(\paperwidth,2cm);
            \node[inner sep=0pt,text=white,font=\large\sffamily,above] (bottominfo) at ([yshift=.7cm]current page.south) {
                \@author\hspace{4cm}\@publishers\hspace{4cm}\today
            };
        \end{pgfonlayer}
        \fill[color=black!50,opacity=.2]node[append after command={
            ([yshift=0.5cm]bookinfo.north west) rectangle ([yshift=-0.5cm]bookinfo.south east)},minimum width=\paperwidth,opacity=1,align=left,inner sep=0pt,anchor=west] (bookinfo) at ([shift={(0,4cm)}]current page.west) {\hspace{-7cm}
                \begin{varwidth}{\linewidth}
                    \setlength\baselineskip{3ex}
                    \textcolor{black!10!white}{\Huge \textbf{\@title}} \\[.6cm]
                    \textcolor{black!10!white}{\Large \@subtitle}
                \end{varwidth}
                };
    \end{tikzpicture}
	\makeatother % 将@重置为非字母字符
	\vspace{0cm} % 标题和子标题的间距
	% 结束左对齐环境
\end{flushleft}
\ClearShipoutPicture % 清除背景图片
\end{titlepage}
% --------------------------------- 主要内容写在下面 --------------------------------- %
\pagestyle{Mainpage} % 页面样式
\chapimg{images/pexels-photo-1452701.jpeg}

\begin{titlepage}
    \newgeometry{left=2cm,right=2cm,top=2.5cm,bottom=2.2cm}
    \tableofcontents
    \restoregeometry
\end{titlepage}
%——------------------------------%

\partimg{images/pexels-photo-931018.jpeg}
\part{Proof of Opennes Conjecture}
\input{chapter1.tex}

\partimg{images/pexels-photo-931018.jpeg}
\part{Proof of Opennes Conjecture}
\input{chapter2.tex}

\partimg{images/pexels-photo-931018.jpeg}
\part{Proof of Opennes Conjecture}
\input{chapter3.tex}

\partimg{images/pexels-photo-931018.jpeg}
\part{Proof of Opennes Conjecture}
\input{chapter4.tex}

%-----------------------------------%
\newgeometry{left=2cm,right=2cm,bottom=2cm,top=3cm}
\printbibliography[heading=bibintoc]


\chapter{Appendix}
\section{拓展链接}
\begin{itemize}
    \item 泛函分析概论:
    
    (\url{https://mp.weixin.qq.com/s/ebhG6fFNry0kqKulTzFxeg})
    \item 
\end{itemize}

\newpage
\section{经典反例}



\newpage
\section{其他}

\begin{figure}[!htbp]
	\centering
		\sidecaption{一些常见的赋范空间及其性质\label{fig:7.1}}{\includegraphics[width=\textwidth]{images/Appendix/一些常见的赋范空间与性质.jpg}}
\end{figure}

\begin{figure}[!htbp]
	\centering
		\sidecaption{一些收敛的定义\label{fig:7.2}}{\includegraphics[width=\textwidth]{images/Appendix/{6CB4C3F6-D97B-4C8B-B9CC-49E00321CF45}.png}}
\end{figure}
\end{document}