\chapter{集成学习}

\section{个体与集成}\label{sec:8.1}
集成学习通过构建并结合多个学习器来完成学习任务,其一般结构为:使用某种策略结合多个“个体学习器”,从而获得比单一学习器更为优越的泛化性能。

如果基分类器的误差相互独立,那么由Hoeffding不等式:
\[
\begin{aligned} P(H(\boldsymbol{x}) \neq f(\boldsymbol{x})) & =\sum_{k=0}^{\lfloor T / 2\rfloor}\binom{T}{k}(1-\epsilon)^{k} \epsilon^{T-k} \\ & \leqslant \exp \left(-\frac{1}{2} T(1-2 \epsilon)^{2}\right)\end{aligned}
\]随着集成个体学习器的数目$T$的增多,误差将指数级下降至接近于0。但实际上,个体学习器来自同一问题,显然不可能互相独立。

如何产生“好且不同”的个体学习器是集成学习研究的核心。
\marginpar{\footnotesize 好对应学习器的准确性,不同代表个体学习器的多样性。}

集成学习大致可分为两类:
\begin{itemize}
    \item 序列化(串行)方法:个体学习器之间存在强依赖关系,必须串行生成,代表方法:Boosting;
    \item 并行化方法:个体学习器之间不存在强依赖关系,可同时生成。代表方法:Bagging,随机森林。
\end{itemize}

\section{Boosting}\label{sec:8.2}
Boosting算法的基本原理是:从初始训练集训练一个基学习器,再根据基学习器的表现对训练样本分布进行调整,使得先前基学习器做错的训练样本在后续受到更多关注,然后基于调整后的样本分布来训练下一个基学习器;如此重复进行,直至基学习器数目达到事先指定的值$T$,最终将这$T$个基学习器进行加权结合。

Boosting算法的代表是Adaboost。

\begin{Keynote}
笔者不认为这个过于复杂的推导会考,但PPT详细讲述了Adaboost的推导过程,因此这边也加入了笔记,但仅供参考。(考了能推出来的算你无敌)
\end{Keynote}

\thm{Adaboost}{thm_ref}{这里基于“加性模型”(基学习器的线性组合)来推导Adaboost算法。

假设我们基于数据集$D$,根据样本权值分布$\mathcal{D}_t$训练得到分类器$h_t$,那么基学习器的线性组合即为\[
H(\boldsymbol{x})=\sum_{t=1}^{T} \alpha_{t} h_{t}(\boldsymbol{x}),
\]而我们的目标则是最小化指数损失函数:\[
\ell_{\exp }(H \mid \mathcal{D})=\mathbb{E}_{\boldsymbol{x} \sim \mathcal{D}}\left[e^{-f(\boldsymbol{x}) H(\boldsymbol{x})}\right]
\]
能使得指数损失函数最小化的$H(x)$对上式的偏导数应为0,于是对其求偏导:\[
\frac{\partial \ell_{\exp }(H \mid \mathcal{D})}{\partial H(\boldsymbol{x})}=-e^{-H(\boldsymbol{x})} P(f(\boldsymbol{x})=1 \mid \boldsymbol{x})+e^{H(\boldsymbol{x})} P(f(\boldsymbol{x})=-1 \mid \boldsymbol{x}) = 0,
\]可以得到:
\[
H(\boldsymbol{x})=\frac{1}{2} \ln \frac{P(f(x)=1 \mid \boldsymbol{x})}{P(f(x)=-1 \mid \boldsymbol{x})},
\]因此,有:
\[
\begin{aligned} \operatorname{sign}(H(\boldsymbol{x})) & =\operatorname{sign}\left(\frac{1}{2} \ln \frac{P(f(x)=1 \mid \boldsymbol{x})}{P(f(x)=-1 \mid \boldsymbol{x})}\right) \\ & =\left\{\begin{array}{ll}1, & P(f(x)=1 \mid \boldsymbol{x})>P(f(x)=-1 \mid \boldsymbol{x}) \\ -1, & P(f(x)=1 \mid \boldsymbol{x})<P(f(x)=-1 \mid \boldsymbol{x})\end{array}\right. \\ & =\underset{y \in\{-1,1\}}{\arg \max } P(f(x)=y \mid \boldsymbol{x}),\end{aligned}
\]
这说明,$\operatorname{sign}(H(\boldsymbol x))$达到了贝叶斯最优错误率,指数损失函数最小化则分类错误率最小化;因此,指数损失函数是分类任务原本0/1损失函数的一致的替代损失函数,应将其作为优化目标。

Adaboost在基于分布$\mathcal{D}_t$迭代生成基分类器$h_t$后,其权重$\alpha_t$应使得$\alpha_t h_t$最小化指数损失函数:\[
\begin{aligned} \ell_{\exp }\left(\alpha_{t} h_{t} \mid \mathcal{D}_{t}\right) & =\mathbb{E}_{\boldsymbol{x} \sim \mathcal{D}_{t}}\left[e^{-f(\boldsymbol{x}) \alpha_{t} h_{t}(\boldsymbol{x})}\right] \\ & =\mathbb{E}_{\boldsymbol{x} \sim \mathcal{D}_{t}}\left[e^{-\alpha_{t}} \mathbb{I}\left(f(\boldsymbol{x})=h_{t}(\boldsymbol{x})\right)+e^{\alpha_{t}} \mathbb{I}\left(f(\boldsymbol{x}) \neq h_{t}(\boldsymbol{x})\right)\right] \\ & =e^{-\alpha_{t}} P_{\boldsymbol{x} \sim \mathcal{D}_{t}}\left(f(\boldsymbol{x})=h_{t}(\boldsymbol{x})\right)+e^{\alpha_{t}} P_{\boldsymbol{x} \sim \mathcal{D}_{t}}\left(f(\boldsymbol{x}) \neq h_{t}(\boldsymbol{x})\right) \\ & =e^{-\alpha_{t}}\left(1-\epsilon_{t}\right)+e^{\alpha_{t}} \epsilon_{t} \quad \epsilon_{t}=P_{\boldsymbol{x} \sim \mathcal{D}_{t}}\left(h_{t}(\boldsymbol{x}) \neq f(\boldsymbol{x})\right)\end{aligned}
\]
上式对权重求偏导并置零可以得到:
\[
    \alpha_t = \frac{1}{2}\ln(\frac{1-\epsilon_t}{\epsilon_t})
\]

Adaboost在获得对应的分布$H_{t-1}$后,样本分布需要调整以生成新的一轮基学习器$h_t$,理想的$h_t$会纠正$H_{t-1}$的所有错误,即最小化$\ell_{exp}(H_{t-1}+h\mid \mathcal{D})$,通过泰勒展开可以得到更新公式:
\[
\begin{aligned} h_{t}(\boldsymbol{x}) & =\underset{h}{\arg \min } \ell_{\exp }\left(H_{t-1}+h \mid \mathcal{D}\right) \\ & =\underset{h}{\arg \min } \mathbb{E}_{\boldsymbol{x} \sim \mathcal{D}}\left[e^{-f(\boldsymbol{x}) H_{t-1}(\boldsymbol{x})}\left(1-f(\boldsymbol{x}) h(\boldsymbol{x})+\frac{1}{2}\right)\right] \\ & =\underset{h}{\arg \max } \mathbb{E}_{\boldsymbol{x} \sim \mathcal{D}}\left[e^{-f(\boldsymbol{x}) H_{t-1}(\boldsymbol{x})} f(\boldsymbol{x}) h(\boldsymbol{x})\right] \\ & =\underset{h}{\arg \max } \mathbb{E}_{\boldsymbol{x} \sim \mathcal{D}}\left[\frac{e^{-f(\boldsymbol{x}) H_{t-1}(\boldsymbol{x})}}{\mathbb{E}_{\boldsymbol{x} \sim \mathcal{D}}\left[e^{-f(\boldsymbol{x}) H_{t-1}(\boldsymbol{x})}\right]} f(\boldsymbol{x}) h(\boldsymbol{x})\right],\end{aligned}
\]令$\mathcal{D_t}$表示分布:\[
\mathcal{D}_{t}(\boldsymbol{x})=\frac{\mathcal{D}(\boldsymbol{x}) e^{-f(\boldsymbol{x}) H_{t-1}(\boldsymbol{x})}}{\mathbb{E}_{\boldsymbol{x} \sim \mathcal{D}}\left[e^{-f(\boldsymbol{x}) H_{t-1}(\boldsymbol{x})}\right]},
\]可以简化上式为\[
h_{t}(\boldsymbol{x})   =\underset{h}{\arg \max }\  \mathbb{E}_{\boldsymbol{x} \sim \mathcal{D}_{t}}[f(\boldsymbol{x}) h(\boldsymbol{x})] .
\]又由$f(x), h(x)\in{-1,+1}$可知:\[
\begin{aligned}
f(\boldsymbol{x}) h(\boldsymbol{x}) & =1-2 \mathbb{I}(f(\boldsymbol{x}) \neq h(\boldsymbol{x})) \\
h_{t}(\boldsymbol{x})&=\underset{h}{\arg \min } \mathbb{E}_{\boldsymbol{x} \sim \mathcal{D}_{t}}[\mathbb{I}(f(\boldsymbol{x}) \neq h(\boldsymbol{x}))]
\end{aligned}
\]
最终样本分布的更新公式为:
\[
\begin{aligned} \mathcal{D}_{t+1}(\boldsymbol{x}) & =\frac{\mathcal{D}(\boldsymbol{x}) e^{-f(\boldsymbol{x}) H_{t}(\boldsymbol{x})}}{\mathbb{E}_{\boldsymbol{x} \sim \mathcal{D}}\left[e^{\left.-f(\boldsymbol{x}) H_{t}(\boldsymbol{x})\right]}\right.} \\ & =\frac{\mathcal{D}(\boldsymbol{x}) e^{-f(\boldsymbol{x}) H_{t-1}(\boldsymbol{x})} e^{-f(\boldsymbol{x}) \alpha_{t} h_{t}(\boldsymbol{x})}}{\mathbb{E}_{\boldsymbol{x} \sim \mathcal{D}}\left[e^{\left.-f(\boldsymbol{x}) H_{t}(\boldsymbol{x})\right]}\right.} \\ & =\mathcal{D}_{t}(\boldsymbol{x}) \cdot e^{-f(\boldsymbol{x}) \alpha_{t} h_{t}(\boldsymbol{x})} \frac{\mathbb{E}_{\boldsymbol{x} \sim \mathcal{D}}\left[e^{\left.-f(\boldsymbol{x}) H_{t-1}(\boldsymbol{x})\right]}\right.}{\mathbb{E}_{\boldsymbol{x} \sim \mathcal{D}}\left[e^{\left.-f(\boldsymbol{x}) H_{t}(\boldsymbol{x})\right]}\right.}\end{aligned}
\]
}
\marginpar{\footnotesize 如果想搞懂上面一串公式究竟在推什么,强烈建议阅读南瓜书第八章(8.5——8.19)。}

Adaboost需要基学习器学习特定的数据分布,这可以通过\textbf{重赋权法}(每次对每个训练样本重新赋予权重)或者\textbf{重采样法}(每次对整个训练集重新采样后用于训练)实现。若使用重采样法,还需要\textbf{重启动机制}防止产生不合格学习器导致算法提前终止,性能下降,

\section{Bagging与随机森林}\label{sec:8.3}
并行化的集成学习可以通过在训练数据集上采样不同的样本子集,再从每个子集训练得到一个基学习器。

\textbf{Bagging}算法给定包括$m$个样本的数据集,有放回的随机采样得到T个含m个样本(可能有重复)的采样集,训练得到基学习器并结合。结合时,对分类任务采用简单投票法,回归任务采用简单平均法。

Bagging的优势在于:时间复杂度很低,可使用包外估计的方式验证泛化性能,能够降低方差,在不剪枝的决策树、神经网络等易受样本影响的学习器上效果更好。

\textbf{随机森林}是bagging的一个扩展,是基于决策树的Bagging集成的一个变种,在决策树的训练过程中引入了随机属性选择。对于基决策树的每个节点,先从该节点的属性结合中随机选择一个包括k个属性的子集,再从这个子集中选择一个最优属性进行划分。
\marginpar{\footnotesize 传统决策树是从所有的属性中选择一个最优属性。}

\section{结合策略}\label{sec:8.4}
学习器的组合可以带来三方面的好处:
\begin{itemize}
    \item 统计方面:结合多个学习器可以减小“多个假设在训练集上性能一致”的风险;
    \item 计算方面:结合多个学习器可以降低局部极小带来的糟糕泛化性能;
    \item 表示方面:结合多个学习器可以提升相应假设空间的大小。
\end{itemize}
\begin{figure}[!htbp]
	\centering
		\sidecaption{学习器结合的好处\label{fig:8.1}}{\includegraphics[width=.8\textwidth]{images/combination.png}
  }
\end{figure}
一些常见的结合策略:
\begin{enumerate}
    \item 平均法:
    常用于数值型输出,包括:
    \begin{enumerate}
        \item 简单平均法:\[H(x) = \frac{1}{T}\sum_{i=1}^T h_i(x)\]
        
        \item 加权平均法:\[
H(\boldsymbol{x})=\sum_{i=1}^{T} w_{i} h_{i}(\boldsymbol{x}), \quad w_{i} \geq 0 
,\quad \sum_{i=1}^{T} w_{i}=1
\]
    \end{enumerate}
    \item 投票法:常用于分类任务,这里的$h_i^j$代表学习器$h_i$在类别$c_j$上的输出:
    \begin{enumerate}
        \item 绝对多数投票法:\[
H(\boldsymbol{x})=\left\{\begin{array}{ll}c_{j} & \text { if } \sum_{i=1}^{T} h_{i}^{j}(\boldsymbol{x})>\frac{1}{2} \sum_{k=1}^{l} \sum_{i=1}^{T} h_{i}^{k}(\boldsymbol{x}) \\ \text { rejection } & \text { otherwise } .\end{array}\right.
\]
        \item 相对多数投票法:\[H(\boldsymbol x) = C_{\underset{j}{\arg \max }\sum_{i=1}^{T}h_i^j(x)}\]即预测为得票最多的标记。
        
        \item 加权投票法:\[H(\boldsymbol x) = C_{\underset{j}{\arg \max}\sum_{i=1}^{T} w_i h_i^j(x)}, \quad w_{i} \geq 0 
,\quad \sum_{i=1}^{T} w_{i}=1\]
    \end{enumerate}
    \item 学习法:使用另一个学习器进行结合,\textbf{Stacking}为典型代表。
    Stacking的优点在于强大的性能,缺点在于次级学习器的输入属性表示和次级学习算法对 Stacking 集成的泛化性能有很大影响。
\end{enumerate}

\section{多样性}\label{sec:8.5}
对于集成的泛化误差而言,我们有公式:
\[
E = \bar{E} - \bar{A},
\]
其中$E$代表集成泛化误差,$\hat{E}$代表所有个体学习器泛化误差的加权均值,$\hat{A}$代表所有个体学习器的加权分歧值。从该式可以得到:个体学习器准确性越高,多样性越大,其集成越好。

常见的增强个体学习器的多样性的方法包括:
\begin{itemize}
    \item \textbf{数据样本扰动}:给定初始数据集,产出不同的数据子集用于产生基学习器,一般是基于采样法,对“不稳定基学习器”(决策树,神经网络)很有效。
    \item \textbf{输入属性扰动}:选择不同的“子空间”(属性子集)训练基学习器,通过减少属性提升训练效率。
    \item \textbf{输出表示扰动}:操纵输出表示以增强多样性,例如:翻转法改变输入样本的标记;输出调剂法将分类输出改为回归输出;ECOC法将多类任务分解为一系列两类任务。
    \item \textbf{算法参数扰动}:随机设置不同的参数,例如用“负相关法”强制个体神经网络使用不同参数。
\end{itemize}

\section{本章往年考试题目}\label{sec:8.6}

本章暂无往年考试题目。