\chapter{Introduction}
\label{cp:introduction}

\textbf{如下为本笔记中所设定的不同样式的标注,特此汇总,以便区
分:}

\section{列表样式类}
%无序
\subsection{无序列表}
\verb|\begin{itemize}|
\begin{itemize}
  \item List entries start with the \verb|\item| command.
  \item Individual entries are indicated with a black dot, a so-called bullet.
  \item The text in the entries may be of any length.
\end{itemize}

%有序
\subsection{有序列表}
\verb|\begin{enumerate}|
\begin{enumerate}%1
    \item First level item
    \item First level item
    \begin{enumerate}%2
        \item Second level item
        \item Second level item
            \begin{enumerate}%3
                \item Third level item
                \item Third level item
                    \begin{enumerate}%4
                        \item Fourth level item
                        \item Fourth level item
                    \end{enumerate}
            \end{enumerate}
    \end{enumerate}
\end{enumerate}


\begin{enumerate}[label=\arabic*)]
    \item 
    \item 
    \item 
\end{enumerate}\verb|\begin{enumerate}[label=\arabic*)]|


%多样式列表
\subsection{多样式列表}
\begin{itemize}
  \item[]  This is my first point,\verb|\item[]|%实心圆点
  \item Another point I want to make,\verb|\item | %无样式点
  \item[!] A point to exclaim something!, \verb|\item[!]|%叹号标注
  \item[$\blacksquare$] Make the point fair and square.,\verb|\item[$\blacksquare$]|%实心方块
\end{itemize}

%description-醒目的列表样式
\begin{description}
    \item[NO.1:] description1
    \item[NO.2:] description2
\end{description}\verb|\begin{description}\item[1:]|

\section{自定义主题框类}

%importantbox-重要框
\begin{importantbox}
    importantbox-重要框
\end{importantbox}\verb|\begin{importantbox}|

%pro-命题
\prp{命题名称}{prp_ref}{proposition environment.}\verb|\prp{命题名称}{pro_ref}{proposition environment.}|

%definition-定义
\begin{prompt}
    prompt-提示词
\end{prompt}\verb|\begin{prompt}|

%Lemma-引理
\lem{引理名称}{lem_ref}{Lemma environment.}\verb|\lem{引理名称}{lem_ref}{Lemma environment.}|

%theorem-定理
\thm{定理样式2名称}{thm_ref}{Theorem environment.}\verb|\thm{定理样式2名称}{thm_ref}{Theorem environment.}|

%Corollary-推论
\cor{推论名称}{cor_ref}{Corollary environment.}\verb|\cor{推论名称}{cor_ref}{Corollary environment.}|

%remark-标注
\begin{remark}
    remark-标注名称
\end{remark}\verb|\begin{remark}|

%Keynote-关键笔记
\begin{Keynote}
    Keynote-关键笔记
\end{Keynote}\verb|\begin{Keynote}|

%expansion-延伸部分
\ext{延伸部分名称}{ext_ref}{Extension environment.}\verb|\ext{延伸部分名称}{ext_ref}{Extension environment.}|

%Example-1
\noindent\textbf{Example:}Example:

\verb|\noindent\textbf{Example:}|

%Example-2
\ex{举例名称}{ex_ref}{Example environment.}\verb|\ex{举例名称}{ex_ref}{Example environment.}|

% MATLAB代码示例
\vspace{.3cm}
\begin{lstlisting}[style=matlab, caption={MATLAB代码示例}]
% MATLAB代码示例
clc;
clear;
...
\end{lstlisting}

\verb|\vspace{.3cm}|

\verb|\begin{lstlisting}[style=matlab, caption={MATLAB代码示例}]|

\section{图片样式}
%图片插入样式
\begin{figure}[!htbp]
	\centering
		\sidecaption{图例注释\label{fig:example1}}{\includegraphics[width=.5\textwidth]{images/pexels-photo-1452701.jpeg}}
\end{figure}

