\documentclass[lang=cn,zihao=5,twoside,fontset=fandol]{textbook}
\usepackage{ctex}
\usepackage{lipsum,zhlipsum}
%\usepackage{fontspec} % 加载 fontspec 包
%\setmainfont{SimSun} % 设置主字体为宋体
% ----------------------------- Math Font ------------------------------------- %
\usepackage{newtxmath}
\DeclareSymbolFont{CMlargesymbols}{OMX}{cmex}{m}{n}
\let\sum\relax
\let\prod\relax
\DeclareMathSymbol{\sum}{\mathop}{CMlargesymbols}{"50}
\DeclareMathSymbol{\prod}{\mathop}{CMlargesymbols}{"51}
%% \infty
\DeclareSymbolFont{symbolsCM}{OMS}{cmsy}{m}{n}
\SetSymbolFont{symbolsCM}{bold}{OMS}{cmsy}{b}{n}
\let\txinfty\infty
\DeclareMathSymbol{\infty}{\mathord}{symbolsCM}{"31}
\definecolor{nuanbai}{HTML}{f5f5f5}
\pagecolor{nuanbai}
% ---------------------------------------------------------------------------- %
\overfullrule=5pt
% \showboxdepth=\maxdimen
% \showboxbreadth=\maxdimen
\tracingonline=1
\tracingoutput=1
\usepackage{extarrows}
\makeatletter
\def\@author{Your Name}
\def\@publishers{Nanjing }
\makeatother
\begin{document}
%----------------------------------------------------------------------------------------
%	标题页信息
%----------------------------------------------------------------------------------------
\title{高级机器学习知识点整理}
\subtitle{\textit{Advanced Machine Learning Notes and Checkpoints}}
\author{Sueno Chiaki 清野千秋}
\date{\zhtoday}
%----------------------------------------------------------------------------------------

%----------------------------------------------------------------------------------------
%	插入自定义标题页
%----------------------------------------------------------------------------------------
\begin{titlepage} % 创建一个新的页面
    %用来将图片从左下角开始平铺整个封面
        \AddToShipoutPicture*{%
        \AtPageLowerLeft{%
            \adjustbox{width=1.1\paperwidth, height=1.5\paperheight, keepaspectratio}{% 强制图片至纸张尺寸,但可能裁切
                \includegraphics{images/pexels-photo-3378916.jpeg}
            }
        }
    }
    \begin{flushleft} % 左对齐环境
        \setlength{\leftskip}{1cm} % 左侧缩进
        \makeatletter % 允许访问带有@字符的内部命令
        % \vspace*{4cm} % 标题距离页面顶端的空白
        % {\color{white}\Huge \textbf{\@title} \par} % 使用前文定义的title作为标题
        % \vspace{1cm} % 标题和子标题的间距
        % {\color{white}\Large \@subtitle \par} % 使用前文定义的subtitle作为子标题
        % \vfill % 作者信息前的垂直填充
        % {\color{white}\large \@author \par} % 作者名
        % {\color{white}\large \@publishers \par} % 出版者
        % {\color{white}\large \today\par} % 日期,默认为当天
            \begin{tikzpicture}[overlay,remember picture]
            \begin{pgfonlayer}{bottom}
                \fill[dblue!10,opacity=0.1] (current page.south west) rectangle ++(\paperwidth,2cm);
                \makeatletter
                \node[inner sep=0pt,text=white,font=\large\sffamily,above] (bottominfo) at ([yshift=.7cm]current page.south) {\@author\hspace{4cm}\@publishers\hspace{4cm}\today
                };
                \makeatother
            \end{pgfonlayer}
            \fill[color=black!50,opacity=.2]node[append after command={
                ([yshift=0.5cm]bookinfo.north west) rectangle ([yshift=-0.5cm]bookinfo.south east)},minimum width=\paperwidth,opacity=1,align=left,inner sep=0pt,anchor=west] (bookinfo) at ([shift={(0,4cm)}]current page.west) {\hspace{-7cm}
                    \begin{varwidth}{\linewidth}
                        \setlength\baselineskip{3ex}
                        \textcolor{black!10!white}{\Huge \textbf{\@title}} \\[.6cm]
                        \textcolor{black!10!white}{\Large \@subtitle}
                    \end{varwidth}
                    };
        \end{tikzpicture}
        \makeatother % 将@重置为非字母字符
        \vspace{0cm} % 标题和子标题的间距
        % 结束左对齐环境
    \end{flushleft}
    \ClearShipoutPicture % 清除背景图片
    \end{titlepage}
% --------------------------------- 主要内容写在下面 --------------------------------- %
\pagestyle{Mainpage} % 页面样式
\chapimg{images/pexels-photo-1452701.jpeg}

\begin{titlepage}
    \newgeometry{left=2cm,right=2cm,top=2.5cm,bottom=2.2cm}
    \tableofcontents
    \restoregeometry
\end{titlepage}

\chapter*{写在前面} 

本笔记是南京大学研究生必修课程《高级机器学习》的复习笔记。

实际上,研究生阶段的同学可能并不会亲自去上《高级机器学习》的课程,而到期末,鸿篇巨制的西瓜书,PPT与公式推导也容易让同学们晕头转向,找不到合适的复习资料,更何况在南大的课程安排中,本课程属于闭卷考试,抱着一大摞PPT和近300页的教材阅读,实在是过于费时费力。

基于如上的思考,笔者决定对现有的高级机器学习资料进行整合,提出一份知识点整理笔记。这份笔记会包括:
\begin{itemize}
    \item 教材与PPT上涉及到的知识点详述;
    \item 一些关于重要公式推导的补充内容;
    \item 笔者所收集到的往年卷考试原题与相关解析。
\end{itemize}

笔者收集到的往年卷为2018-2023六年的考题,其中2021年的考题由于回忆不全,只包含一部分题目。根据2025年春季学期的考纲,高级机器学习的期末考试为10-11道题,每题10分,包括1道附加题,满分100,每章大概会出1道题。

这份复习笔记的诞生离不开南大学长学姐们公开资料的支援,在此感谢:
\begin{itemize}
    \item Maxwell Lyu 吕导的AML复习提纲(22年)(\url{https://blog.xlworkshop.ltd/post/nju-aml-2022/})
    \item Peanuts 学长/学姐的2022/2023年AML原题与解析(\url{https://zhuanlan.zhihu.com/p/702835146})
\end{itemize}

值得一提的是,囿于笔者水平以及写作时间不足,这份笔记不可避免地会出现考点的疏漏,病句错字,格式混乱等各种问题,还请各位多多包涵。

复习笔记基于Latex编写,在Github上开源,因此,希望贡献新的考试题目或者更正笔者出现的错误的各位,欢迎提交Pull Request,为这份笔记继续丰富内容。

只愿这份笔记能够帮助到所有南大CS研究生同学们,大家一起认真考试,投身科研。

\begin{flushright}
	2025.06.05

    清野千秋

    于常州楼作
\end{flushright}
\chapter{Introduction}
\label{cp:introduction}

\textbf{如下为本笔记中所设定的不同样式的标注,特此汇总,以便区
分:}

\section{列表样式类}
%无序
\subsection{无序列表}
\verb|\begin{itemize}|
\begin{itemize}
  \item List entries start with the \verb|\item| command.
  \item Individual entries are indicated with a black dot, a so-called bullet.
  \item The text in the entries may be of any length.
\end{itemize}

%有序
\subsection{有序列表}
\verb|\begin{enumerate}|
\begin{enumerate}%1
    \item First level item
    \item First level item
    \begin{enumerate}%2
        \item Second level item
        \item Second level item
            \begin{enumerate}%3
                \item Third level item
                \item Third level item
                    \begin{enumerate}%4
                        \item Fourth level item
                        \item Fourth level item
                    \end{enumerate}
            \end{enumerate}
    \end{enumerate}
\end{enumerate}


\begin{enumerate}[label=\arabic*)]
    \item 
    \item 
    \item 
\end{enumerate}\verb|\begin{enumerate}[label=\arabic*)]|


%多样式列表
\subsection{多样式列表}
\begin{itemize}
  \item[]  This is my first point,\verb|\item[]|%实心圆点
  \item Another point I want to make,\verb|\item | %无样式点
  \item[!] A point to exclaim something!, \verb|\item[!]|%叹号标注
  \item[$\blacksquare$] Make the point fair and square.,\verb|\item[$\blacksquare$]|%实心方块
\end{itemize}

%description-醒目的列表样式
\begin{description}
    \item[NO.1:] description1
    \item[NO.2:] description2
\end{description}\verb|\begin{description}\item[1:]|

\section{自定义主题框类}

%importantbox-重要框
\begin{importantbox}
    importantbox-重要框
\end{importantbox}\verb|\begin{importantbox}|

%pro-命题
\prp{命题名称}{prp_ref}{proposition environment.}\verb|\prp{命题名称}{pro_ref}{proposition environment.}|

%definition-定义
\begin{prompt}
    prompt-提示词
\end{prompt}\verb|\begin{prompt}|

%Lemma-引理
\lem{引理名称}{lem_ref}{Lemma environment.}\verb|\lem{引理名称}{lem_ref}{Lemma environment.}|

%theorem-定理
\thm{定理样式2名称}{thm_ref}{Theorem environment.}\verb|\thm{定理样式2名称}{thm_ref}{Theorem environment.}|

%Corollary-推论
\cor{推论名称}{cor_ref}{Corollary environment.}\verb|\cor{推论名称}{cor_ref}{Corollary environment.}|

%remark-标注
\begin{remark}
    remark-标注名称
\end{remark}\verb|\begin{remark}|

%Keynote-关键笔记
\begin{Keynote}
    Keynote-关键笔记
\end{Keynote}\verb|\begin{Keynote}|

%expansion-延伸部分
\ext{延伸部分名称}{ext_ref}{Extension environment.}\verb|\ext{延伸部分名称}{ext_ref}{Extension environment.}|

%Example-1
\noindent\textbf{Example:}Example:

\verb|\noindent\textbf{Example:}|

%Example-2
\ex{举例名称}{ex_ref}{Example environment.}\verb|\ex{举例名称}{ex_ref}{Example environment.}|

% MATLAB代码示例
\vspace{.3cm}
\begin{lstlisting}[style=matlab, caption={MATLAB代码示例}]
% MATLAB代码示例
clc;
clear;
...
\end{lstlisting}

\verb|\vspace{.3cm}|

\verb|\begin{lstlisting}[style=matlab, caption={MATLAB代码示例}]|

\section{图片样式}
%图片插入样式
\begin{figure}[!htbp]
	\centering
		\sidecaption{图例注释\label{fig:example1}}{\includegraphics[width=.5\textwidth]{images/pexels-photo-1452701.jpeg}}
\end{figure}


%---------------------------------------------------------------------------- %

\chapter{绪论}
\section{机器学习是什么?}\label{sec:1.1}
\begin{Keynote}
机器利用\textbf{数据}学习人类的\textbf{经验},从而不断提升系统自身\textbf{性能}的过程。
\end{Keynote}
一个典型的机器学习过程:

有类别标记的训练数据 -> 学习算法训练 -> 模型 <- 类别未知的新数据样本

\section{机器学习的历史与未来?}\label{sec:1.2}

机器学习的历史:
\begin{itemize}
 \item 推理期(60年代——80年代初期)
 \item 知识期(80年代初期——90年代中期)
 \item 学习期(90年代中期——今)
\end{itemize}
机器学习的未来:\textbf{可信稳健协同机器学习}
\section{机器学习与其他领域的关系?}\label{sec:1.3}
与数据挖掘领域:ML是数据挖掘的关键支撑技术,但ML更偏技术,DM更偏应用。

与数据科学领域:ML是数据科学实现智能化的关键步骤,没有ML分析数据,大数据无法利用。

与计算机视觉/自然语言处理领域:ML是CV/NLP的核心技术。

与神经科学领域:ML的发展过程中经常受到神经科学的启发。
\section{机器学习前沿进展}\label{sec:1.4}
学术期刊:
\begin{itemize}
\item AIJ 《Artificial Intelligence》
\item JMLR 《Journal of Machine Learning Research》
\item TPAMI《IEEE Trans. on Pattern Analysis and Machine Intelligence》
\item TKDE《IEEE Trans. on Knowledge and Data Engineering》
\item MLJ 《Machine Learning》
\item TNNLS 《IEEE Trans. on Neural Network and Learning Systems
\end{itemize}
学术会议:
\begin{itemize}
\item ICML (International Conference on Machine Learning)
\item NeurIPS (Neural Information Processing Systems)
\item KDD (ACMSIGKDD Conf. on Knowledge Discovery and Data Mining)
\item AAAI (AAAI conference on Artificial Intelligence)
\item IJCAI (International Joint Conference on Artificial Intelligence)
\item ICLR (International Conference on Learning Representation)
\item MLA (中国机器学习及其应用研讨会)
\end{itemize}

\section{本章往年考试题目}\label{sec:1.5}
\ex{2021年考试原题}{ex_ref}{简述机器学习的鲁棒性?}
鲁棒性是机器学习性能的关键。
\begin{itemize}
 \item 对人类用户错误鲁棒
 \item 对网络攻击鲁棒
 \item 对错误目标鲁棒
 \item 对不正确模型鲁棒
 \item 对未建模现象鲁棒
\end{itemize}

\ex{2022/2023年考试原题}{ex_ref}{什么是机器学习?举出两个应用实例。机器学习的未来挑战和发展?}
\textbf{定义}:机器利用\textbf{数据}学习人类的\textbf{经验},从而不断提升系统自身\textbf{性能}的过程。

应用实例略,未来发展略。

\textbf{挑战}:稳健性不足。


%\part{模型评估与选择}
\chapter{模型评估与选择}

\section{经验误差与过拟合}\label{sec:2.1}

\begin{itemize}
    \item 泛化误差:在“未来样本”(unseen instance)上的误差。
    \item 经验误差:在训练集合上的误差,也被称之为“训练误差”。
    \item 过拟合:把特性当成共性,特征学习过度。
    \item 欠拟合:把共性当成特性,特征学习不足。
\end{itemize}

\begin{figure}[!htbp]
	\centering
		\sidecaption{一个实际过拟合/欠拟合的例子\label{fig:example1}}{\includegraphics[width=.7\textwidth]{images/over-and-under.png}}
\end{figure}
%--------------------------------%
%--------------------------------%
\section{评估方法}\label{sec:2.2}
\begin{itemize}
    \item 留出法:将数据集划分为两个互斥的集合——训练与测试集,并尽可能保持数据分布以及类别比例的一致性;
    \item 交叉验证法:将数据集分层采样划分为K个大小相似的互斥子集,每次用k-1个子集的并集作为训练集,余下的子集作为测试集,最终返回测试结果的均值。
    \item 留一法:k=数据集的样本数。
\end{itemize}

上述方法从上到下变得越精确,但复杂度也更高,计算开销更大。

%--------------------------------%
%--------------------------------%
\section{性能度量}\label{sec:2.3}
性能度量是衡量模型泛化能力的评价标准,反映了任务需求;使用不同的性能度量往往会导致不同的评判结果。

\prp{性能度量}{pro_ref}{对于\textbf{回归任务},最常用的是均方误差MSE:
\[
E(f ; D)=\frac{1}{m} \sum_{i=1}^{m}\left(f\left(\boldsymbol{x}_{i}\right)-y_{i}\right)^{2}
\]

对于\textbf{分类任务},错误率与精度是最常用的性能度量:
\begin{enumerate}
    \item 分类错误率:$ E(f ; D)=\frac{1}{m} \sum_{i=1}^{m} \mathbb{I}\left(f\left(\boldsymbol{x}_{i}\right) \neq y_{i}\right) $
    \item 精度Accuracy:$\operatorname{acc}(f ; D) =\frac{1}{m} \sum_{i=1}^{m} \mathbb{I}\left(f\left(\boldsymbol{x}_{i}\right)=y_{i}\right) = 1-E(f ; D) . $
\end{enumerate}

除此之外,基于分类结果混淆矩阵,还有: 
\begin{itemize}
    \item 查准率:$ P = \frac{TP}{TP+FP} $
    \item 查全率:$ R = \frac{TP}{TP+FN} $
    \item F1度量(F1-Score):$ F_1 = \frac{2\times P\times R}{P+R} = \frac{2\times TP}{样例总数+TP-TN} $
    \item F-\beta 度量:$ F_\beta = \frac{(1+\beta^2)\times P\times R}{(\beta^2\times P)+R} $
\end{itemize}

以及AUC(Area Under Curve),基于真比例率(y)与假正例率(x)形成的ROC曲线,曲线下面积大小即为AUC:
\[
AUC = \frac{1}{2}\sum_{i=1}^{m-1}(x_{i+1}-x_i)\cdot (y_i+ y_{i+1})
\]
}

\begin{figure}[!htbp]
	\centering
		\sidecaption{分类结果混淆矩阵,上述公式中对应的值参见此处。\label{fig:example1}}{\includegraphics[width=.6\textwidth]{images/types.png}}
\end{figure}

%--------------------------------%
%--------------------------------%
\section{比较检验}\label{sec:2.4}
测试性能不等于泛化性能,且会随着测试集的变化而变化,机器学习算法本身有一定的随机性。因此,需要利用\textbf{假设检验}为学习器的性能比较提供数学理论依据。

\thm{二项检验}{thm_ref}{考虑假设“$\epsilon \le \epsilon_0$”, 则在$1 - \alpha$的概率内所能观测到的最大错误率如下式计算:
\[
\bar{\epsilon} = \min \epsilon \ \text{s.t.} \ \sum_{i=\epsilon\times m+1}^m\binom{m}{i}\epsilon_0^i(1-\epsilon_0)^{m-i} \lt \alpha
\]

此时,若测试错误率$\hat{\epsilon}$小于临界值$\bar{\epsilon}$,则我们称,在\alpha 的显著度下,假设不能被拒绝,泛化器的错误率不大于$\epsilon_0$。否则,该假设可被拒绝,即在\alpha 的显著度下可认为学习器的泛化错误率大于$\epsilon_0$.
}

\thm{T-检验}{thm_ref}{很多时候,我们并非只做一次留出法统计,而是通过多次留出法或是使用交叉验证法进行多次训练测试,得到k个测试错误率$\hat{\epsilon}_{1}, \hat{\epsilon}_{2}, \ldots, \hat{\epsilon}_{k}$,则平均错误率$\mu$与方差$\sigma^2$为:
\[
\begin{array}{c}\mu=\frac{1}{k} \sum_{i=1}^{k} \hat{\epsilon}_{i} \\ \sigma^{2}=\frac{1}{k-1} \sum_{i=1}^{k}\left(\hat{\epsilon}_{i}-\mu\right)^{2}\end{array}
\]
这k个测试错误率可看作泛化错误率$\epsilon_0$的独立采样,则变量
\[
\tau_{t}=\frac{\sqrt{k}\left(\mu-\epsilon_{0}\right)}{\sigma}
\]
服从自由度为$k - 1$的t分布。
因此,假设$\epsilon = \epsilon_0$对于显著度\alpha, 若$ [t_{-\alpha/2}, t_{\alpha/2}] $ 位于临界范围$ | \mu - \epsilon_0 | $内,则假设不能被拒绝,即可认为泛化错误率$\epsilon = \epsilon_0$,其置信度为$1 - \alpha$。
}

对两个学习器A和B,若我们使用k折交叉验证法得到的测试错误率分
别为 $ \epsilon_{1}^{A}, \ldots, \epsilon_{k}^{A} $ 和 $ \epsilon_{1}^{B}, \ldots, \epsilon_{k}^{B} $,其中$ \epsilon_i^A$和$\epsilon_i^B$是在相同的第i折训练/测试集上得到的结果,则可用k折交叉验证"成对t检验"来进行比较检验。其基本思想是:若两个学习器的性能相同,则\textbf{它们使用相同的训练/测试集得到的测试错误率}应相同,即$\epsilon_i^A=\epsilon_i^B$。


%--------------------------------%
%--------------------------------%
\section{偏差与方差}\label{sec:2.5}
对于回归任务,泛化误差可通过“偏差——方差分解”拆解为:
\begin{figure}[!htbp]
	\centering
	{\includegraphics[width=\textwidth]{images/bias-var.png}}
\end{figure}

\begin{Keynote}
泛化性能是由\textbf{学习算法的能力},\textbf{数据的充分性}以及\textbf{学习任务本身的难度}共同决定。
\end{Keynote}

\section{本章往年考试题目}\label{sec:2.6}
\ex{2023年考试原题}{ex_ref}{从训练误差和泛化误差的角度阐述过拟合和欠拟合的含义,并举出1-2个机器学习模型例子说明如何克服过拟合和欠拟合。}
\begin{itemize}
    \item 过拟合:训练误差小,泛化误差大
    \item 欠拟合:训练误差和泛化误差都大
\end{itemize}

克服过拟合的方法:
\begin{itemize}
    \item 模型简单化
    \item 增加训练数据
    \item 增加正则化约束
    \item 提前结束训练
\end{itemize}

克服欠拟合的方法:
\begin{itemize}
    \item 模型复杂化
    \item 加入新的特征
    \item 降低正则化约束
\end{itemize}

\ex{2022年考试原题}{ex_ref}{理想模型为什么难以实现?评估方法,性能度量,比较检验分别解决什么问题?}
理想模型难以实现的原因:数据的可获得性和质量问题;模型的复杂性和计算资源的问题;泛化能力的局限性;人工智能的局限性;建模复杂性和不确定性。
\begin{itemize}
    \item 评估方法:解决的是“如何获取测试结果”;
    \item 性能度量:解决的是“如何评估性能优劣”;
    \item 比较检验:解决的是“如何判断实质差别”。
\end{itemize}


%\part{线性模型}
\chapter{线性模型}
\begin{importantbox}
从本章开始,一系列公式可能会要求推导。限于作者精力,复习讲义不提供完整推导,仅提供部分参考步骤与提示。推荐查看南瓜书《机器学习公式详解》以获得对应推导的详细过程。
\end{importantbox}
\section{基本思想}\label{sec:2.1}
线性模型的基本思想是,对于一个由$d$个属性描述的示例$\boldsymbol{x}=\left(x_{1} ; x_{2} ; \ldots ; x_{d}\right)$,学习得到一个属性的线性组合用于预测的函数,即
\[
f(\boldsymbol{x})=w_{1} x_{1}+w_{2} x_{2}+\ldots+w_{d} x_{d}+b
\]
一般用向量形式写作:
\[
f(\boldsymbol{x})=\boldsymbol{w}^{\mathrm{T}} \boldsymbol{x}+b, 
\boldsymbol{w}=\left(w_{1} ; w_{2} ; \ldots ; w_{d}\right)
\]
\marginpar{\footnotesize 线性模型中的w直观表达了各属性在预测中的重要性,使线性模型具有较好的可解释性。}
线性模型的问题在于,如何确定$w$与$b$的值,使得线性模型拥有良好的性能。

\section{线性回归任务}\label{sec:3.2}
在最简单的情形(即输入只有一个属性的情况下),线性回归试图习得\[
f(x_i)=wx_i+b,\ s.t. \ f(x_i) \simeq y_i
\]
线性回归通过均方误差最小化确定w与b,这种基于均方误差最小化来求解模型的方法被称为\textbf{最小二乘法}。

\thm{最小二乘法}{thm_ref}{\[
\begin{aligned}\left(w^{*}, b^{*}\right) & =\underset{(w, b)}{\arg \min } \sum_{i=1}^{m}\left(f\left(x_{i}\right)-y_{i}\right)^{2} \\ & =\underset{(w, b)}{\arg \min } \sum_{i=1}^{m}\left(y_{i}-w x_{i}-b\right)^{2}\end{aligned}
\]
为使上式$E_{(w,b)}=\sum_{i=1}^m(y_i-wx_i-b)^2$对应的值最小化,将$E_{(w,b)}$分别对w与b求导得到:
\[
\begin{aligned}\frac{\partial E_{(w, b)}}{\partial w} & =2\left(w \sum_{i=1}^{m} x_{i}^{2}-\sum_{i=1}^{m}\left(y_{i}-b\right) x_{i}\right) \\ \frac{\partial E_{(w, b)}}{\partial b}& =2\left(m b-\sum_{i=1}^{m}\left(y_{i}-w x_{i}\right)\right)\end{aligned}
\]
令上述两个公式等于零,可得到w与b的闭式解。
\[
\begin{aligned} w &=\frac{\sum_{i=1}^{m} y_{i}\left(x_{i}-\bar{x}\right)}{\sum_{i=1}^{m} x_{i}^{2}-\frac{1}{m}\left(\sum_{i=1}^{m} x_{i}\right)^{2}} \\ b&=\frac{1}{m} \sum_{i=1}^{m}\left(y_{i}-w x_{i}\right)\end{aligned}
\]
}
\marginpar{\footnotesize 上述推导的提示:先得到b的闭式解,再代入w的公式内。}

更一般的情形是,样本由$d$个属性描述,此时我们试图学习的是\[
f\left(\boldsymbol{x}_{i}\right)=\boldsymbol{w}^{\mathrm{T}} \boldsymbol{x}_{i}+b,\ s.t. \ 
f\left(\boldsymbol{x}_{i}\right) \simeq y_{i}
\]
这被称之为多元线性回归。

多元线性回归的做法是将$w$与$b$吸收入向量形式,得到一个统一向量用于后续计算:$\hat{\boldsymbol{w}} = (\boldsymbol{w};b)=(w_1;...;w_d;b)\in \mathbb{R}^{(d+1)\times 1}$,同时增加一个表示形式$\hat{\boldsymbol{x}} =(x_1;...;x_d;1)\in \mathbb{R}^{(d+1)\times 1}$,将数据集D表示为一个矩阵:
\[\boldsymbol{X} = \left(\begin{array}{ccccc}x_{11} & x_{12} & \cdots & x_{1 d} & 1 \\ x_{21} & x_{22} & \cdots & x_{2 d} & 1 \\ \vdots & \vdots & \ddots & \vdots & \vdots \\ x_{m 1} & x_{m 2} & \cdots & x_{m d} & 1\end{array}\right)=\left(\begin{array}{cc}\boldsymbol{x}_{1}^{\mathrm{T}} & 1 \\ \boldsymbol{x}_{2}^{\mathrm{T}} & 1 \\ \vdots & \vdots \\ \boldsymbol{x}_{m}^{\mathrm{T}} & 1\end{array}\right) = \left[\begin{array}{c}\hat{\boldsymbol{x}}_{1}^{\mathrm{T}} \\ \vdots \\ \hat{\boldsymbol{x}}_{m}^{\mathrm{T}}\end{array}\right] \in \mathbb{R}^{m \times (d+1)}\]
\marginpar{\footnotesize 推导的提示:对简单最小二乘法中的$wx_i+b$整体代换为$\boldsymbol{w}^Tx_i+b$,然后替换为向量形式,改写为向量内积并提取$X$。}
于是类似最小二乘法,我们可以得到:\[
\hat{\boldsymbol{w}}^{*}=\underset{\hat{\boldsymbol{w}}}{\arg \min }(\boldsymbol{y}-\mathbf{X} \hat{\boldsymbol{w}})^{\mathrm{T}}(\boldsymbol{y}-\mathbf{X} \hat{\boldsymbol{w}})
\]
\marginpar{\footnotesize 推导的提示:展开并使用矩阵微分公式$ \frac{\partial \boldsymbol{a}^{\mathrm{T}} \boldsymbol{x}}{\partial \boldsymbol{x}}=\frac{\partial \boldsymbol{x}^{\mathrm{T}} \boldsymbol{a}}{\partial \boldsymbol{x}}=\boldsymbol{a}, \frac{\partial \boldsymbol{x}^{\mathrm{T}} \mathbf{A} \boldsymbol{x}}{\partial \boldsymbol{x}}=\left(\mathbf{A}+\mathbf{A}^{\mathrm{T}}\right) \boldsymbol{x} $。}
令$ E_{\hat{\boldsymbol{w}}}=(\boldsymbol{y}-\mathbf{X} \hat{\boldsymbol{w}})^{\mathrm{T}}(\boldsymbol{y}-\mathbf{X} \hat{\boldsymbol{w}}) $,对$\hat{\boldsymbol{w}}$求导可得:
\[
\frac{\partial E_{\hat{\boldsymbol{w}}}}{\partial \hat{\boldsymbol{w}}}=2 \mathbf{X}^{\mathrm{T}}(\mathbf{X} \hat{\boldsymbol{w}}-\boldsymbol{y})
\]

令上式为0,并根据矩阵的特征可得到闭式解。
\begin{itemize}
    \item $\mathbf{X}^T\mathbf{X}$满秩或正定,则$ \hat{\boldsymbol{w}}^{*}=\left(\mathbf{X}^{\mathbf{T}} \mathbf{X}\right)^{-1} \mathbf{X}^{\mathrm{T}} \boldsymbol{y} $,$ f\left(\hat{\boldsymbol{x}}_{i}\right)=\hat{\boldsymbol{x}}_{i}^{\mathrm{T}}\left(\mathbf{X}^{\mathrm{T}} \mathbf{X}\right)^{-1} \mathbf{X}^{\mathrm{T}} \boldsymbol{y} $。
    \item 若不正定,则引入对应的正则化项。
\end{itemize}

\section{二分类任务}\label{sec:3.3}
对于二分类任务而言,其输出标记$y\in \{0,1\}$,需要寻找将线性模型$z = \boldsymbol{w}^\mathrm{T}\boldsymbol{x}+b$的实值转换为0/1值的函数。

\begin{itemize}
    \item 单位阶跃函数:$ y=\left\{\begin{aligned} 0, & z<0 \\ 0.5, & z=0 \\ 1, & z>0\end{aligned}\right. $
    但实际上该函数不连续不可导。
    \item 对数几率函数:\[
        y=\frac{1}{1+e^{-z}}\]
    该函数任意阶可导,将其带入广义线性模型公式$y = g^{-1}(\boldsymbol{w}^\mathrm{T}\boldsymbol{x}+b)$可得:
    \[
    y=\frac{1}{1+e^{-(\boldsymbol{w}^\mathrm{T}\boldsymbol{x}+b)}}\]
\end{itemize}

可以认为$y$是样本$\boldsymbol{x}$为正例的可能性,$1–y$为样本为负例的可能性,两者的比值$\frac{y}{1-y}$即为几率,反应了样本为正的相对可能性。
对上述公式取对数可以得到:
\[
\ln\frac{y}{1-y} = \boldsymbol{w}^\mathrm{T}\boldsymbol{x}+b
\]

可以发现实际上是在用线性回归模型结果预测对数概率,因此该模型称为“对数几率回归”。

\marginpar{\footnotesize 下列推导的提示:简写两个条件概率,将$p(y_i| \boldsymbol x_i; \boldsymbol w_i, b)$带入对数似然中,再利用$y_i$取值仅为0或1简化。}

\thm{极大似然法}{thm_ref}{对于对数几率回归而言,我们一般通过将$y$视为后验概率估计$p(y=1|x)$的方式重写上式:\[
\ln \frac{p(y=1 \mid \boldsymbol{x})}{p(y=0 \mid \boldsymbol{x})}=\boldsymbol{w}^{\mathrm{T}} \boldsymbol{x}+b
\]
\[
\begin{aligned}p(y=1 \mid \boldsymbol{x})=\frac{e^{\boldsymbol{w}^{\mathrm{T}} \boldsymbol{x}+b}}{1+e^{\boldsymbol{w}^{\mathrm{T}} \boldsymbol{x}+b}} \\ p(y=0 \mid \boldsymbol{x})=\frac{1}{1+e^{\boldsymbol{w}^{\mathrm{T}} \boldsymbol{x}+b}} \end{aligned}
\]
因此,对于给定数据集$\{(x_i,y_i)\}_{i=1}^{m}$,我们最大化对数似然:
\[
\ell(\boldsymbol{w}, b)=\sum_{i=1}^{m} \ln p\left(y_{i} \mid \boldsymbol{x}_{i} ; \boldsymbol{w}, b\right)
\]
令$ \boldsymbol{\beta}=(\boldsymbol{w} ; b), \hat{\boldsymbol{x}}=(\boldsymbol{x} ; 1) $,于是 $ \boldsymbol{w}^\mathrm{T}\boldsymbol{x}+b $简写为$\boldsymbol{\beta}^\mathrm{T}\hat{\boldsymbol x}$。
再令$ p_{1}\left(\hat{\boldsymbol{x}}_{i} ; \boldsymbol{\beta}\right)=p(y=1 \mid \hat{\boldsymbol{x}} ; \boldsymbol{\beta}), \  p_{0}\left(\hat{\boldsymbol{x}}_{i} ; \boldsymbol{\beta}\right)=p(y=0 \mid \hat{\boldsymbol{x}} ; \boldsymbol{\beta})=1-p_{1}\left(\hat{\boldsymbol{x}}_{i} ; \boldsymbol{\beta}\right) $,可得\[
p\left(y_{i} \mid \boldsymbol{x}_{i} ; \boldsymbol{w}_{i}, b\right)=y_{i} p_{1}\left(\hat{\boldsymbol{x}}_{i} ; \boldsymbol{\beta}\right)+\left(1-y_{i}\right) p_{0}\left(\hat{\boldsymbol{x}}_{i} ; \boldsymbol{\beta}\right)
\]
\[
\ell(\boldsymbol{\beta})=\sum_{i=1}^{m}\left(-y_{i} \boldsymbol{\beta}^{\mathrm{T}} \hat{\boldsymbol{x}}_{i}+\ln \left(1+e^{\beta^{\mathrm{T}} \hat{\boldsymbol{x}}_{i}}\right)\right)
\]
使用牛顿法迭代求解该高阶可导连续函数,得到:
\[
\boldsymbol{\beta}^{*}=\underset{\boldsymbol{\beta}}{\arg \min } \ell(\boldsymbol{\beta})
\]
第t+1轮迭代解的更新公式:
\[
\boldsymbol{\beta}^{t+1}=\boldsymbol{\beta}^{t}-\left(\frac{\partial^{2} \ell(\boldsymbol{\beta})}{\partial \boldsymbol{\beta} \partial \boldsymbol{\beta}^{\mathrm{T}}}\right)^{-1} \frac{\partial \ell(\boldsymbol{\beta})}{\partial \boldsymbol{\beta}}
\]
}

\section{多分类任务}\label{sec:3.4}

考虑N个类别$C_1,C_2,...,C_N$,多分类任务的基本思路是“拆解法”,将任务拆解为若干个二分类任务进行求解,为拆解子任务训练分类器并集成得到多分类结果。
给定数据集$
D= \{ (\boldsymbol x_1,y_1),(\boldsymbol x_2,y_2),... (\boldsymbol x_m,y_m) \}, y_i\in\{C_1, C_2,...,C_N \}
$,
\begin{itemize}
    \item 一对一OvO:将所有类别两两配对,生成$frac{N(N-1)}{2}$个二类任务与对应的学习分类器。在测试阶段,得到$frac{N(N-1)}{2}$个分类结果,投票取出现次数最多的结果作为最终类别。训练时间短但存储和测试开销大。
    \item 一对其余OvR:每次将一个类作为正例,其他所有类作为反例训练N个分类器。测试时若只有一个分类器的输出为正类,则对应类别标记作为最终分类结果,否则使用置信度最大的类别作为最终类别。训练时间长但存储测试开销小。
    \item 多对多:每次将若干个类作为正类,若干个其他类别作为反类。需要对正反类的选择做特殊设计。一种框架为纠错输出码,通过对N个类别做M次划分得到类别长度为M的编码,测试时计算M个分类器组成的编码与已存在编码之间的距离得到类别。一般来说,编码越长,纠错能力越强。
\end{itemize}

\begin{figure}[!htbp]
	\centering
		\sidecaption{一对一与一对其余的示例\label{fig:4.1}}{\includegraphics[width=.9\textwidth]{images/OVOVR.png}}
\end{figure}

\begin{figure}[!htbp]
	\centering
		\sidecaption{纠错输出码的示例。海明距离 = 欧氏距离的平方/4\label{fig:4.2}}{\includegraphics[width=.8\textwidth]{images/ECOC.png}}
\end{figure}

\section{线性模型总结}\label{sec:3.5}
线性模型的优点是:形式简单,易于建模,具有一定的可解释性。

线性模型的缺点是:难以处理非线性问题。

线性模型是许多非线性模型的基础,如神经网络(层级结构)或支持向量机(高维映射)。

\section{本章往年考试题目}\label{sec:3.6}

\ex{2021年考试原题}{ex_ref}{推导最小二乘法。}
过程略,详见\ref{sec:3.2}节。

\ex{2022年考试原题}{ex_ref}{最小二乘法的限制?}
\begin{itemize}
    \item 对数据的噪声敏感
    \item 无法处理不平衡数据
    \item 无法处理线性不可分的数据
    \item 不适用于高维稀疏数据
    \item 缺乏概率输出,在一些场景下无法满足需求
\end{itemize}

\ex{2023年考试原题}{ex_ref}{推导多元线性回归的闭式解,并详述线性模型的优缺点。}
见\ref{sec:3.2}节与\ref{sec:3.5}节。

%\part{支持向量机}
\chapter{支持向量机}

\section{间隔与支持向量}\label{sec:4.1}

给定一个数据集$
D=\left\{\left(\boldsymbol{x}_{1}, y_{1}\right),\left(\boldsymbol{x}_{2}, y_{2}\right), \ldots,\left(\boldsymbol{x}_{m}, y_{m}\right)\right\}, y_{i} \in\{-1,+1\}$,分类学习希望在样本空间内找到一个划分超平面,将不同类别的样本分开,最好是找到两类训练样本“正中间”的划分超平面。


这个划分超平面$(\boldsymbol w, b )$可通过如下线性方程描述:\[
\boldsymbol{w}^\mathrm{T}\boldsymbol{x}+b = 0
\]其中$ \boldsymbol{w}=\left(w_{1} ; w_{2} ; \ldots ; w_{d}\right) $代表超平面的方向,b为位移项,代表了超平面与原点之间的距离。

样本空间内任一点$\boldsymbol x$到超平面之间的距离为:\[
r=\frac{\left|\boldsymbol{w}^{\mathrm{T}} \boldsymbol{x}+b\right|}{\|\boldsymbol{w}\|}
\]

\marginpar{\footnotesize 最中间的超平面对于训练样本的随机扰动容忍性最高,分类结果最鲁棒,泛化能力最强。}

假设超平面$(w, b)$ 能将训练样本正确分类,即对于 $ (\boldsymbol x_i, y_i) \in D $,若 $ y_i = +1 $,则有 $ \boldsymbol w^T \boldsymbol x_i + b > 0 $; 若 $ y_i = -1 $,则有 $ \boldsymbol w^T \boldsymbol x_i + b < 0 $。令
\[
\begin{cases}
w^T x_i + b \geq +1, & y_i = +1; \\
w^T x_i + b \leq -1, & y_i = -1.
\end{cases}
\]

\begin{figure}[!htbp]
	\centering
		\sidecaption{支持向量与间隔。\label{fig:4.1}}{\includegraphics[width=.6\textwidth]{images/support.png}}
\end{figure}

如图所示,\textbf{距离超平面最近}的这几个训练样本点使上式的等号成立,它们被称为“支持向量”(support vector),两个异类支持向量到超平面的距离之和为
\[
\gamma = \frac{2}{\|\boldsymbol w\|}
\]
它被称为“间隔”(margin)。

\section{支持向量机基本型}\label{sec:4.2}

欲找到具有“最大间隔”(maximum margin)的划分超平面,也就是要找到能满足式 (6.3) 中约束的参数 $\boldsymbol w$ 和 $b$,使得 $\gamma$ 最大,即
\marginnote{\footnotesize 这里突然出现的$y_i$是样本点的+/-,实际上就是样本点$(x_i,y_i), y\in \{+1, -1\}$}
\[
\max_{\boldsymbol w, b} \frac{2}{\|\boldsymbol w\|} \quad \text{s.t.}\quad y_i(\boldsymbol w^T \boldsymbol x_i + b) \geq 1, i = 1, 2, \ldots, m.
\]

最大化间隔只需要最大化$||\boldsymbol w||^{-1}$,也即最小化$||\boldsymbol w||^{2}$,于是重写为:

\[
\min_{\boldsymbol w, b} \frac{1}{2}\| \boldsymbol w\|^2 \quad \text{s.t.}\quad y_i(\boldsymbol w^T \boldsymbol x_i + b) \geq 1, i = 1, 2, \ldots, m.
\]

上式就是基本型的优化目标。

\section{支持向量机对偶型}\label{sec:4.3}
对基本型的每条约束引入拉格朗日乘子$\alpha_i \ge 0$能够得到其拉格朗日函数:\[
L(\boldsymbol{w}, b, \boldsymbol{\alpha})=\frac{1}{2}\|\boldsymbol{w}\|^{2}+\sum_{i=1}^{m} \alpha_{i}\left(1-y_{i}\left(\boldsymbol{w}^{T} \boldsymbol{x}_{i}+b\right)\right)
\]

令$L(\boldsymbol{w}, b, \boldsymbol{\alpha})$对$\boldsymbol w, b$的偏导数为零可得:
\[
\boldsymbol{w}=\sum_{i=1}^{m} \alpha_{i} y_{i} \boldsymbol{x}_{i}, \quad 0=\sum_{i=1}^{m} \alpha_{i} y_{i}
\]
将其回代可得到基本型的对偶问题,即对偶型的优化目标:
\[
\max_{\alpha} \sum_{i=1}^m\alpha_i - \frac{1}{2}\sum_{i=1}^{m}\sum_{j=1}^m \alpha_i\alpha_j y_i y_j\boldsymbol x_i^{\mathrm{T}}\boldsymbol x_j \]
\[\quad \text{s.t.}\sum_{i=1}^m\alpha_iy_i = 0, \alpha_i \ge 0, i= 1, 2, \ldots, m.
\]

\section{特征空间映射}\label{sec:4.4}

若不存在一个能正确划分两类样本的超平面,则将样本从原始空间映射到一个更高维的特征空间,使得样本在这个特征空间内线性可分。

如果原始空间是有限维的,则一定存在一个高维特征空间使得样本可分。

\section{核函数}\label{sec:4.5}
令 $\phi(x)$ 表示将 $x$ 映射到高维空间后的特征向量,于是,在特征空间中划分超平面所对应的模型可表示为
\[
f(x) = w^T \phi(x) + b
\]
其中 $w$ 和 $b$ 是模型参数。类似基本型,有
\[
\min_{\boldsymbol w, b} \frac{1}{2} \|\boldsymbol w\|^2\quad \text{s.t.} \quad y_i \big(\boldsymbol w^T \phi(\boldsymbol x_i) + b\big) \geq 1, \quad i = 1, 2, \ldots, m.
\]

其对偶问题是
\[
\max_{\alpha} \sum_{i=1}^m \alpha_i - \frac{1}{2} \sum_{i=1}^m \sum_{j=1}^m \alpha_i \alpha_j y_i y_j \phi(\boldsymbol x_i)^T \phi(\boldsymbol x_j).
\]
\[
\text{s.t.} \quad \sum_{i=1}^m\alpha_iy_i = 0, \alpha_i \ge 0, i= 1, 2, \ldots, m.
\]

计算映射到高维空间的内积可能非常复杂,因此我们设计\textbf{核函数}以绕过显式考虑特征映射以及计算高维内积的困难。

\thm{核函数}{thm_ref}{核函数是这样的一个函数:\[
\kappa\left(\boldsymbol{x}_{i}, \boldsymbol{x}_{j}\right)=\left\langle\phi\left(\boldsymbol{x}_{i}\right), \phi\left(\boldsymbol{x}_{j}\right)\right\rangle=\phi\left(\boldsymbol{x}_{i}\right)^{\mathrm{T}} \phi\left(\boldsymbol{x}_{j}\right)
\]
该函数表示:$\boldsymbol x_i, \boldsymbol x_j$在特征空间的内积等于其在原始样本空间中通过该函数$\kappa$计算得到的结果。通过将其带入对偶式,我们可以得到\textbf{支持向量展式}:\[
\begin{aligned} f(\boldsymbol{x}) & =\boldsymbol{w}^{\mathrm{T}} \phi(\boldsymbol{x})+b \\ & =\sum_{i=1}^{m} \alpha_{i} y_{i} \phi\left(\boldsymbol{x}_{i}\right)^{\mathrm{T}} \phi(\boldsymbol{x})+b \\ & =\sum_{i=1}^{m} \alpha_{i} y_{i} \kappa\left(\boldsymbol{x}, \boldsymbol{x}_{i}\right)+b\end{aligned}
\]
由Mercer定理,只要一个对称函数所对应的核矩阵\textbf{半正定},即可作为核函数使用。
}

\begin{figure}[!htbp]
	\centering
		\sidecaption{一些常用的核函数,高斯核分类性能很强。\label{fig:4.2}}{\includegraphics[width=.8\textwidth]{images/kernel.png}}
\end{figure}

\section{软间隔支持向量机}\label{sec:4.6}
实际应用中,很难确定合适的核函数使得训练样本在特征空间中线性可分。因此引入“软间隔”,允许向量机在一定的样本上出错,不满足约束$y_i(\boldsymbol{w}^\mathrm{T}\boldsymbol{x}+b) \ge 1$。

软间隔的优化目标为:
\[
\min _{\boldsymbol{w}, b} \frac{1}{2}\|\boldsymbol{w}\|^{2}+C \sum_{i=1}^{m} \ell_{0 / 1}\left(y_{i}\left(\boldsymbol{w}^{\mathrm{T}} \boldsymbol{x}_{i}+b\right)-1\right)
\]
其中$\ell_{0 / 1}$是0/1损失函数:\[
\ell_{0 / 1}(z)=\left\{\begin{array}{ll}1, & \text { if } z<0 \\ 0, & \text { otherwise }\end{array}\right.
\]
但0/1损失函数性质不好,不连续不可导,于是使用替代函数hinge损失:$\ell_{hinge}(z) = \max(0, 1-z)$与松弛变量$\xi_i \ge 0$,得到软间隔SVM基本型:

\[
\min_{\boldsymbol w, b, \xi_i} \frac{1}{2}\| \boldsymbol w\|^2+C\sum_{i=1}^{m}\xi_i \quad \text{s.t.}\quad y_i(\boldsymbol w^T \boldsymbol x_i + b) \geq 1-\xi_i, \ \xi_i\ge 0, \ i = 1, 2, \ldots, m.
\]

同样,引入拉格朗日算子可得对应拉格朗日函数:
\[
L(\boldsymbol{w}, b, \boldsymbol{\alpha}, \boldsymbol \xi , \boldsymbol \mu)=\frac{1}{2}\|\boldsymbol{w}\|^{2}+ C\sum_{i=1}^{m}\xi_i +\sum_{i=1}^{m} \alpha_{i}\left(1- \xi_i - y_{i}\left(\boldsymbol{w}^{T} \boldsymbol{x}_{i}+b\right)\right) - \sum_{i=1}^m\mu_i\xi_i
\]
令$L(\boldsymbol{w}, b, \boldsymbol{\alpha}, \boldsymbol \xi , \boldsymbol \mu)$对$\boldsymbol w, b, \xi_i$的偏导数为零可得:
\[
\boldsymbol{w}=\sum_{i=1}^{m} \alpha_{i} y_{i} \boldsymbol{x}_{i}, \quad 0=\sum_{i=1}^{m} \alpha_{i} y_{i}, \quad C = \alpha_i+\mu_i
\]

代回原式可得:

\[
\max_{\alpha} \sum_{i=1}^m\alpha_i - \frac{1}{2}\sum_{i=1}^{m}\sum_{j=1}^m \alpha_i\alpha_j y_i y_j\boldsymbol x_i^{\mathrm{T}}\boldsymbol x_j \]
\[\quad \text{s.t.}\sum_{i=1}^m\alpha_iy_i = 0, 0 \le \alpha_i \le C, i= 1, 2, \ldots, m.
\]

\section{正则化}\label{sec:4.7}

写出SVM的一般形式,这个形式也被称之为正则化问题:
\[
\min _{f} \Omega(f)+C \sum_{i=1}^{m} \ell\left(f\left(\boldsymbol{x}_{i}\right), y_{i}\right)
\]
前项$\Omega(f)$被称为结构风险,用于描述模型本身的某些性质,也被称为正则化项;后项$l(f(\boldsymbol x_i), y_i)$称为经验风险,用于描述模型与训练数据的契合程度;C被称为正则化常数,用于在两者之间调整平衡。

常用的正则化项包括$L_0, L_1, L_2$范数,前两者倾向于非零分量少而稀疏,后者倾向于取值均衡(非零分量个数尽量稠密)。

\section{本章往年考试题目}\label{sec:4.8}

\ex{2022年考试原题}{ex_ref}{SVM基本型和对偶型的联系是什么?它们对应哪些情况,优势和原因?}
\textbf{联系}:

基本型和对偶型都是二次凸规划问题,根据凸优化理论,都可以利用成熟数值优化软件包得到全局最优解。

基本型善于处理:低维数据/高维稀疏数据

对偶型擅长处理:高维稠密数据/非线性数据(吸收核函数)

\textbf{原因:}

基本型直接优化超平面参数,没有对偶问题的转换,参数较少,易于理解和实现。

对偶型则转化为二次凸优化问题,且容易引入核函数技巧,将难以切分的数据映射到高维后实现数据分类。

\ex{2023/2025年考试原题}{ex_ref}{写出SVM基本型的优化目标,并推导至对偶型,引入核函数,说明核函数的好处。}

见\ref{sec:4.2}, \ref{sec:4.3}, \ref{sec:4.5}节。

核函数的优点: 应对非线性数据,提高模型泛化能力,简化计算过程,并为不同类型的数据提供多样性。

%\part{神经网络}
\chapter{神经网络}
\section{神经网络的发展史}\label{sec:5.1}
\begin{itemize}
    \item 1943——1963,以感知机为代表。
    \item 1969年,单层神经网络不能解决非线性问题,进入低谷期。
    \item 1982——2000,以Hopfield网络与反向传播算法为代表。
    \item 2006——至今,预训练+微调使深度网络最优化更简单。
\end{itemize}
\section{神经元模型}\label{sec:5.2}
神经元模型的基本思想:“电位”超过“阈值”被“激活”,向其他神经元发送信号。

经典模型:\textbf{M-P神经元模型}
\begin{figure}[!htbp]
	\centering
		\sidecaption{MP神经元模型。基于输入信号与带权重连接与神经元阈值的比较,通过激活函数得到输出\label{fig:5.1}}{\includegraphics[width=.8\textwidth]{images/MPNeuron.png}}
\end{figure}

理想的神经元模型激活函数是阶跃函数,但不连续不可导不光滑影响实际使用,因此一般使用\textbf{sigmoid函数}进行代替,后者平滑且无穷阶可导。

\begin{figure}[!htbp]
	\centering
		\sidecaption{典型的神经元激活函数\label{fig:5.2}}{\includegraphics[width=.8\textwidth]{images/sigmoid.png}}
\end{figure}

\section{感知机与多层网络}\label{sec:5.3}

感知机由两层神经元组成,输入曾接受外界输入信号传递给输出层,输出层是M-P神经元,也称为“阈值逻辑单元”。

\marginpar{\footnotesize 非线性可分意味着不能通过线性超平面划分模式,这就是第一次低谷出现的原因,可以想想上一节的SVM。}

感知机可以容易的实现逻辑与或非运算这种\textbf{线性可分问题},但局限于仅有输出层一层逻辑处理单元,学习能力有限,无法处理非线性可分问题。

为解决这一问题,我们构建\textbf{多层感知机},在输出层与输出层之间加入一层具有激活函数的功能神经元,称之为隐层或隐藏层。

更一般的神经网络被称之为“\textbf{多层前馈神经网络}”,其中相邻两层的神经元全互联,不存在同层以及跨层连接;输入层神经元接收外界输入,隐含层与输出层神经元对信号进行加工输出,学习过程为根据训练数据调整神经元的“连接权”以及功能神经元的“阈值”。

\section{误差逆传播算法}\label{sec:5.4}
误差逆传播算法也被称之为\textbf{后向传播(BackPropagation, BP)}算法,是最成功以及最广泛使用的神经网络学习算法。

给定一个训练集$D = \{(\boldsymbol x_1, \boldsymbol y_1), ..., (\boldsymbol x_m, \boldsymbol y_m)\}, \boldsymbol x_i\in \mathbb{R}^d, \boldsymbol y_i\in \mathbb{R}^l$,即输入示例由$d$个属性描述,输出$l$维实值向量。

\marginpar{\footnotesize 推导的提示:Sigmoid函数有个很好的性质,$f'(x)=f(x)(1-f(x))$。}

我们给出一个$d$个输入神经元,$q$个隐层神经元与$l$个输出神经元的多层前向前馈网络结构,模拟对该训练集的实际学习,如下图,并假设其中所有功能神经元使用sigmoid函数作为激活函数。

\begin{figure}[!htbp]
	\centering
		\sidecaption{BP网络的示意图\label{fig:5.3}}{\includegraphics[width=.8\textwidth]{images/BP.png}}
\end{figure}

\thm{后向传播推导}{thm_ref}{首先声明记号:$\theta_j$代表输出层的第j个神经元阈值,$\gamma_h$代表隐层第h个神经元阈值,$v_{ih}$代表输入层与隐层神经元之间的连接权重,$w_{hj}$代表隐层与输出层神经元之间的连接权重。

因此,网络中需要优化的参数为:$(d+l)q(所有的连接权重) + (l+q)(功能神经元的阈值)$

对于样例$(\boldsymbol x_k, \boldsymbol y_k)$,假设网络的实际输出为$\hat{\boldsymbol y}_k$,则:

隐层第h个神经元接收到的输入为$\alpha_h = \sum_{i=1}^d v_{ih}x_i$,得到的输出为$b_h= f(\alpha_h-\gamma_h)$;

输出层第j个神经元接收到的输入为$\beta_j = \sum_{i=q}^dw_{hj}b_h$,输出为$\hat{y}_j^k = f(\beta_j-\theta_j)$。

网络在$(\boldsymbol x_k, \boldsymbol y_k)$上的均方误差即为\[E_k = \frac{1}{2}\sum_{j=1}^l (\hat{y}_j^k-y_j^k)^2.\]

BP算法基于梯度下降策略,以误差率为目标,计算负梯度方向对参数进行调整。给定学习率$\eta$,可以计算参数更新式,以$v_{ih}$为例:
\[
\begin{aligned}
\Delta v_{i h} & =-\eta \frac{\partial E_{k}}{\partial v_{i h}} \\
& = -\eta \frac{\partial E_{k}}{\partial b_h} \cdot \frac{\partial b_h}{\partial \alpha_{h}}  \cdot \frac{\partial \alpha_{h}}{\partial v_{i h}} \\ 
& = -\eta \frac{\partial E_{k}}{\partial b_h} \cdot \frac{\partial b_h}{\partial \alpha_{h}} \cdot x_i\\
g_j &= -\frac{\partial E_{k}}{\partial \hat{y}_j^k} \cdot \frac{\partial \hat{y}_j^k}{\partial \beta_{j}}\\
& = -(\hat{y}_j^k- y_j^k)f'(\beta_j-\theta_j)\\
&= \hat{y}_j^k(1-\hat{y}_j^k)(y_j^k-\hat{y}_j^k)\\
e_h & = -\frac{\partial E_{k}}{\partial b_h} \cdot \frac{\partial b_h}{\partial \alpha_{h}} \\ 
& = -\sum_{j=1}^l\frac{\partial E_{k}}{\partial \beta_j}\cdot \frac{\partial \beta_j}{\partial b_{h}}f'(\alpha_h-\gamma_h) \\
& = \sum_{j=1}^{l}w_{hj}g_jf'(\alpha_h-\gamma_h) \\
& = b_n(1-b_n)\sum_{j=1}^{l}w_{hj}g_j  \\
\Delta v_{ih} & = \eta e_hx_i
\end{aligned}
\]
同理,可以计算图中每个参数的变化量,并反向传播更新这些参数。
}

在实际应用中,存在标准BP算法以及累计BP算法。标准BP算法每次对单个训练样例更新权值与阈值;而累计BP算法读取整个训练集之后再更新参数,优化整个训练集上的累计误差。

多层前馈网络具有\textbf{强大的学习能力}:包含足够多神经元的隐层, 多层前馈神经网络能以任意精度逼近任意复杂度的连续函数。

\marginpar{\footnotesize 缓解过拟合的方法:Early Stop,正则化。}
但多层前馈网络由于其强大学习能力,非常容易过拟合,且难以估计隐层神经元个数。

\section{全局最小与局部最小}\label{sec:5.5}

学习过程本质上是一个参数寻优过程,在参数空间中,可能存在多个局部最小,但只有一个全局最小,如何找到这个全局最小是一个问题。

“跳出”局部极小一般使用的算法:设置不同的初始参数,模拟退火,随机扰动,遗传算法etc...

\section{深度学习}\label{sec:5.6}
深度学习是具有很多个隐层的神经网络,计算能力的提升与训练数据的大幅增加使得复杂模型的训练成为可能。

增加模型复杂度的方式:
\begin{itemize}
    \item 模型宽度:增加隐层神经元的数目;
    \item 模型深度:增加隐层的数目(更有效)。
\end{itemize}

在多个隐层内,复杂模型的反向传播可能发生梯度消失问题,需要别的训练方法。
\begin{itemize}
    \item 预训练+微调:每次训练时将上一层隐层结点的输出作为输入, 本层隐结点的输出作为输出,仅训练一层网络。预训练全部完成后,对整个网络进行微调训练,可以看作对参数分组找到局部最优后再全局调优。
    \item 权共享:每组神经元使用相同的连接权值,e.g. CNN可以用BP进行训练,但在训练中,无论是卷积层还是采样层,每一组神经元都是用相同的连接权,从而大幅减少了需要训练的参数数目。
\end{itemize}

\section{本章往年考试题目}\label{sec:5.7}

\ex{2022/2023/2025年考试原题}{ex_ref}{多层前馈网络的计算能力、局限以及解决方法?}
见\ref{sec:5.4}节.
解决方法:

早停(Early Stop):在训练过程中,若训练误差降低,但是验证误差升高,则停止训练。

正则化:在误差目标函数中增加一项描述网络复杂程度的成分,防止模型过于复杂,例如连接权值和阈值的平方和。

%\part{决策树}
\chapter{决策树}

\section{决策树的基本流程}\label{sec:6.1}
决策树的基本策略是:\textbf{“分而治之”}。

决策树从根节点开始,不断地寻找“划分”属性,将数据集递归地不断划分为小集合。可以说,决策树是一个从根到叶的递归过程,每个中间节点对应一个属性测试,该递归的终止条件为:

\begin{itemize}
    \item 当前节点包含的样本已经全部属于同一类别,\textbf{无需划分};
    \item 当前\textbf{属性集为空},或所有样本在所有属性上取值相同,\textbf{无法划分};这种情况下,生成一个叶节点,类别为该节点所含样本最多的类别。
    \item 当前节点的\textbf{样本集为空,不能划分}。这种情况下,生成一个叶节点,类别为父节点所含样本最多的类别。
\end{itemize}
\section{划分选择}\label{sec:6.2}
决策树学习的关键在于\textbf{选择最优的划分标准}。我们希望决策树的分支节点所包含的样本尽可能属于同一类别,即\textbf{“纯度”}越高越好。

\subsection{信息增益}
信息增益的划分选择依赖于信息熵,信息熵(information entropy) 是度量样本集合纯度最常用的一种指标。假定当前样本集合$D$中第$k$类样本所占的比例为$p_k (k = 1,2,. . . , |\mathcal{Y}|)$,则D的信息熵定义为:
\[
\operatorname{Ent}(D)=-\sum_{k=1}^{\mid \mathcal{Y |}} p_{k} \log _{2} p_{k}
\]
$\operatorname{Ent}(D)$值越小,$D$的纯度越高。

假设某个属性$a$有V个可能的取值$\{a^1,a^2,...,a^V\}$,在决策树上就会产生V个分支节点,第v个节点包括了所有在a上取值为$a^v$的样本,记作$D^v$,我们可以计算出节点的信息熵,并且为其乘上一个系数,该系数由该取值对应的样本数决定,被称之为“信息增益”。
\[
\operatorname{Gain}(D, a)=\operatorname{Ent}(D)-\sum_{v=1}^{V} \frac{\left|D^{v}\right|}{|D|} \operatorname{Ent}\left(D^{v}\right)
\]
\marginpar{\footnotesize 信息增益偏好取值数多的属性。}
信息增益越大,用属性a进行划分得到的纯度提升越大。

\subsection{增益率}
增益率为平衡属性偏好的不利影响,引入“增益率”选择最优划分属性。
\marginpar{\footnotesize 增益率偏好取值数较少的属性。}
\[
\operatorname{Gain\_ ratio}(D, a)=\frac{\operatorname{Gain}(D, a)}{\operatorname{IV}(a)} = \frac{\operatorname{Gain}(D,a)}{-\sum_{v=1}^{V}\frac{\left|D^{v}\right|}{|D|} \log _{2} \frac{\left|D^{v}\right|}{|D|}}
\]

\section{剪枝处理}\label{sec:6.3}
决策树的决策分支过多,可能导致将训练集自身的一些特性当作所有数据均具有的一般性质,从而导致过拟合。因此,需要对决策树进行剪枝从而提升泛化性。

测试泛化性能是否提升的方法:留出法。(留出一部分数据进行验证)

剪枝处理有两种测试思路:
\begin{itemize}
    \item 预剪枝:预剪枝的思路是\textbf{边建树边剪枝},在划分前先估计该节点的划分是否能提升泛化性能(计算验证集精度),若不能提升精度则不允许划分。

    其优点在于能够降低过拟合风险,减少训练与测试时间开销,但有可能导致欠拟合。

    \item 后剪枝:后剪枝的思路是\textbf{先建树后剪枝},考察建树之后每个节点替换为叶节点是否能提升验证集精度,可以则剪枝缩点。

    其优点在于能够降低欠拟合风险,泛化性能更强,但其训练时间开销大。
\end{itemize}

\section{多变量决策树}\label{sec:6.4}

单变量决策树的分类边界是与坐标轴平行的,而多变量决策树则是对属性的线性组合,每个节点都是一个形如$\sum_{i=1}^d w_i a_i=t$的线性分类器,其中$ w_i$是属性$a_i$的权值,$w_i$与$t$可在该节点所含的样本集与属性集上学得。

\section{本章往年考试题目}\label{sec:6.5}

\ex{2022年考试原题}{ex_ref}{决策树为何容易过拟合?解决方案是什么?}

详见\ref{sec:6.3}节。

\ex{2023/2025年考试原题}{ex_ref}{决策树最优划分的两个准则,以及他们在属性选择上的偏好是什么?}

详见\ref{sec:6.2}节。


%\part{贝叶斯分类}
\chapter{贝叶斯分类器}

\section{贝叶斯决策论}\label{sec:7.1}
\textbf{贝叶斯决策论}是概率框架下实施决策的基本方法,对于分类任务,在所有相关的概率都已知的情况下,贝叶斯决策论考虑如何基于这些概率和误判损失来选择最优的类别标记。

假设有N种可能的类别标记,即$\mathcal{Y} = \{c_1, c_2, ...,c_N \}$,令$\lambda _{i j}$代表将标记为$c_j$的样本误分类到$c_i$所产生的损失,则基于后验概率将样本$x$分类到第$i$类的“条件风险”为:
\[
R\left(c_{i} \mid \boldsymbol{x}\right)=\sum_{j=1}^{N} \lambda_{i j} P\left(c_{j} \mid x\right)
\]

我们希望寻找一个判定准则$h:\mathcal{X\to Y}$以最小化整体条件风险:
\[
R(h) = \mathbb{E}_ {\boldsymbol x}[R(h(\boldsymbol x) \mid\boldsymbol x)],
\]
而\textbf{贝叶斯判定准则}表明,为最小化总体风险,只需要在每个样本上选择哪个能使条件风险$R(c\mid\boldsymbol x)$最小的类别标记,即\[
h^{*}(\boldsymbol{x})=\underset{c \in \mathcal{Y}}{\arg \min } R(c \mid \boldsymbol{x})
\]

此时的$h^*$被称为\textbf{贝叶斯最优分类器},该总体风险称为\textbf{贝叶斯风险},这反映了学习性能的理论上限。

但不难发现,贝叶斯判定准则想要使用,就需要获取后验概率$P(c\mid x)$,而在现实任务中后验概率很难获得。因此,机器学习需要基于有限的训练样本集尽可能准确的估计出后验概率$P(c\mid \boldsymbol x)$。

对此,有两种策略:判别式模型或生成式模型。\textbf{判别式模型}通过对给定的$\boldsymbol{x}$直接建模$P(c\mid \boldsymbol x)$来预测$c$,这种模型的代表为决策树/SVM/BP神经网络。而\textbf{生成式模型}则先对联合概率分布$P(\boldsymbol x, c)$建模,再由此获得$P(c|\boldsymbol x)$,其代表为贝叶斯分类器。

\section{朴素贝叶斯与拉普拉斯修正}\label{sec:7.2}
对于生成式模型而言,必然需要考虑的问题为:\marginpar{\footnotesize 使用贝叶斯公式做了拆解。}
\[
P(c \mid \boldsymbol{x})=\frac{P(\boldsymbol{x},c)}{P(\boldsymbol{x})} = \frac{P(c)\ P(\boldsymbol{x}\mid c)}{P(\boldsymbol{x})}
\]
估计$P(c \mid x)$的问题可以转化为基于训练数据D来估计先验$P(c)$与似然$P(\boldsymbol x\mid c)$,但$P(\boldsymbol x\mid c)$也是所有属性上的联合概率,很难从有限的训练样本上估计得到。

朴素贝叶斯分类器采用\textbf{属性条件独立性假设}:即对于已知类别,假设所有属性之间相互独立。于是可以改写上式为:
\[
P(c \mid \boldsymbol{x})=\frac{P(c)\  P(\boldsymbol{x} \mid c)}{P(\boldsymbol{x})}=\frac{P(c)}{P(\boldsymbol{x})} \prod_{i=1}^{d} P\left(x_{i} \mid c\right),
\]
其中d为属性数目,$x_i$为$\boldsymbol x$在第i个属性上的取值。
由于对于所有类别$P(\boldsymbol x)$一致,因此有:
\[
h_{n b}(\boldsymbol{x})=\underset{c \in \mathcal{Y}}{\arg \max } P(c) \prod_{i=1}^{d} P\left(x_{i} \mid c\right)
\]

令$D_c$表示训练集D中第c类样本的集合,则可估计出类先验概率:
\[
    P(c) = \frac{| D_c|}{|D|}
\]

对于离散属性而言,令$D_{c,x_i}$表示$D_c$中在第i个属性上取值为$x_i$的样本组成的集合,则有:
\[
    P(x_i\mid c) =\frac{|D_{c,x_i}|}{|D_c|}
\]
对于连续属性而言,假设概率密度函数$p(x_i\mid c) \sim \mathcal{N}(\mu_{c,i},\sigma^2_{c,i})$,则有:
\[
p\left(x_{i} \mid c\right)=\frac{1}{\sqrt{2 \pi} \sigma_{c, i}} \exp \left(-\frac{\left(x_{i}-\mu_{c, i}\right)^{2}}{2 \sigma_{c, i}^{2}}\right)
\]

为了避免其他属性携带的信息被训练集中未出现的属性值抹去,在估计概率值时通常要使用\textbf{拉普拉斯修正}进行平滑,我们修正类先验概率与离散属性条件概率为:
\[
    \hat{P}(c) = \frac{|D_c|+1}{|D|+N}
\]
\[
    \hat{P}(x_i\mid c) = \frac{|D_{c,x_i}|+1}{|D_c|+N_i}
\]
其中$N$代表训练集中可能的类别数,$N_i$表示第$i$个属性可能的取值数。

\section{半朴素贝叶斯}\label{sec:7.3}
朴素贝叶斯的属性独立性假设在现实中往往难以成立,因此适当考虑一部分属性之间的相互依赖信息的\textbf{半朴素贝叶斯分类器}诞生。

半朴素贝叶斯分类器最常用的策略为\textbf{独依赖估计},该策略假设每个属性在类别之外最多仅依赖一个其他属性,被称为父属性。\[
P(c \mid x) \propto P(c) \prod_{i=1}^{d} P\left(x_{i} \mid c, p a_{i}\right),
\]这其中$pa_i$就是$x_i$的父属性。

如何确定每个属性的父属性有一系列的方法,包括:
\begin{itemize}
    \item SPODE:假设所有属性均依赖同一个属性,这个属性被称为超父Super Father,通过交叉验证等方式确定该属性。
    \item TAN:以属性间的条件”互信息”为边的权重,构建完全图,再利用最大带权生成树算法,仅保留强相关属性间的依赖性。
    \item AODE:尝试以每个属性作为超父构建SPODE,然后集成所有有足够数据支撑的SPODE作为最终结果\[
P(c \mid \boldsymbol{x}) \propto \sum_{\substack{i=1 \\\left|D_{x_{i}}\right| \geqslant m^{\prime}}}^{d} P\left(c, x_{i}\right) \prod_{j=1}^{d} P\left(x_{j} \mid c, x_{i}\right)
\]
\[
\hat{P}\left(c, x_{i}\right)=\frac{\left|D_{c, x_{i}}\right|+1}{|D|+N_{i}}, \quad \hat{P}\left(x_{j} \mid c, x_{i}\right)=\frac{\left|D_{c, x_{i}, x_{j}}\right|+1}{\left|D_{c, x_{i}}\right|+N_{j}}
\]其中$D_{x_i}$是在第i个属性上取值为$x_i$的样本集合,$D_{c,x_i,x_j}$表示类别为c且在第i/j个属性上取值为$x_i$与$x_j$的样本集合。
\end{itemize}

\section{贝叶斯网}\label{sec:7.4}
贝叶斯网通过借助有向无环图DAG构建属性之间的依赖关系。

具体而言,一个贝叶斯网$B=\langle G, \Theta\rangle$由DAG网络结构$G$与直接依赖关系的条件概率集合$\theta_{x_i|\pi_i} = P_B(x_i\mid \pi_i)$组成,其中$\pi_i$是属性$x_i$在G中的父节点集合。给定父结点集,贝叶斯网假设每个属性与其非后裔属性独立。

\begin{figure}[!htbp]
	\centering
		\sidecaption{西瓜问题的一种贝叶斯网结构。\label{fig:7.1}}{\includegraphics[width=.8\textwidth]{images/bayes.png}}
\end{figure}

对于上图而言,其联合概率分布定义为:
\[
P\left(x_{1}, x_{2}, x_{3}, x_{4}, x_{5}\right)=P\left(x_{1}\right) P\left(x_{2}\right) P\left(x_{3} \mid x_{1}\right) P\left(x_{4} \mid x_{1}, x_{2}\right) P\left(x_{5} \mid x_{2}\right)
\]

可知,$x_3,x_4$在给定$x_1$的条件下独立,记作条件独立性$x_{3} \perp x_{4} \mid x_{1}$;给定$x_4$的情况下,$x_1,x_2$必不独立,而若$x_4$未知,则$x_1,x_2$独立,记作边际独立性。三种典型的依赖关系如下图:

\begin{figure}[!htbp]
	\centering
		\sidecaption{三变量之间的典型依赖关系。\label{fig:7.1}}{\includegraphics[width=.6\textwidth]{images/depend.png}}
\end{figure}

将有向图转变为无向图,并在所有V型结构上补充一条边,可以将其转化为道德图,道德图可以导出完整的条件独立性关系:若去掉$z$,$x,y$在图上被分为两个联通分支,则$x \perp y \mid z$。

得到条件独立性关系后,估计出条件概率表,得到最终网络。

\section{本章往年考试题目}\label{sec:7.5}

\ex{2018年考试原题}{ex_ref}{什么是贝叶斯最优分类器?并说明什么是贝叶斯风险。}
见\ref{sec:7.1}节。


\ex{2019, 2020, 2022, 2023,2025年考试原题}{ex_ref}{生成式模型与判别式模型的区别是什么?并各举出一个例子。}
\textbf{判别式模型}通过对给定的$\boldsymbol x$直接建模后验概率$P(c\mid \boldsymbol x)$来预测类别$c$。这一类模型的例子包括决策树,支持向量机与后向传播模型。

\textbf{生成式模型}则先对联合概率分布$P(\boldsymbol x, c)$建模,再由此获得$P(c|\boldsymbol x) = \frac{P(x,c)}{P(x)}$。这一类模型的例子为贝叶斯分类器。

%\part{集成学习}
\chapter{集成学习}

\section{个体与集成}\label{sec:8.1}
集成学习通过构建并结合多个学习器来完成学习任务,其一般结构为:使用某种策略结合多个“个体学习器”,从而获得比单一学习器更为优越的泛化性能。

如果基分类器的误差相互独立,那么由Hoeffding不等式:
\[
\begin{aligned} P(H(\boldsymbol{x}) \neq f(\boldsymbol{x})) & =\sum_{k=0}^{\lfloor T / 2\rfloor}\binom{T}{k}(1-\epsilon)^{k} \epsilon^{T-k} \\ & \leqslant \exp \left(-\frac{1}{2} T(1-2 \epsilon)^{2}\right)\end{aligned}
\]随着集成个体学习器的数目$T$的增多,误差将指数级下降至接近于0。但实际上,个体学习器来自同一问题,显然不可能互相独立。

如何产生“好且不同”的个体学习器是集成学习研究的核心。
\marginpar{\footnotesize 好对应学习器的准确性,不同代表个体学习器的多样性。}

集成学习大致可分为两类:
\begin{itemize}
    \item 序列化(串行)方法:个体学习器之间存在强依赖关系,必须串行生成,代表方法:Boosting;
    \item 并行化方法:个体学习器之间不存在强依赖关系,可同时生成。代表方法:Bagging,随机森林。
\end{itemize}

\section{Boosting}\label{sec:8.2}
Boosting算法的基本原理是:从初始训练集训练一个基学习器,再根据基学习器的表现对训练样本分布进行调整,使得先前基学习器做错的训练样本在后续受到更多关注,然后基于调整后的样本分布来训练下一个基学习器;如此重复进行,直至基学习器数目达到事先指定的值$T$,最终将这$T$个基学习器进行加权结合。

Boosting算法的代表是Adaboost。

\begin{Keynote}
笔者不认为这个过于复杂的推导会考,但PPT详细讲述了Adaboost的推导过程,因此这边也加入了笔记,但仅供参考。(考了能推出来的算你无敌)
\end{Keynote}

\thm{Adaboost}{thm_ref}{这里基于“加性模型”(基学习器的线性组合)来推导Adaboost算法。

假设我们基于数据集$D$,根据样本权值分布$\mathcal{D}_t$训练得到分类器$h_t$,那么基学习器的线性组合即为\[
H(\boldsymbol{x})=\sum_{t=1}^{T} \alpha_{t} h_{t}(\boldsymbol{x}),
\]而我们的目标则是最小化指数损失函数:\[
\ell_{\exp }(H \mid \mathcal{D})=\mathbb{E}_{\boldsymbol{x} \sim \mathcal{D}}\left[e^{-f(\boldsymbol{x}) H(\boldsymbol{x})}\right]
\]
能使得指数损失函数最小化的$H(x)$对上式的偏导数应为0,于是对其求偏导:\[
\frac{\partial \ell_{\exp }(H \mid \mathcal{D})}{\partial H(\boldsymbol{x})}=-e^{-H(\boldsymbol{x})} P(f(\boldsymbol{x})=1 \mid \boldsymbol{x})+e^{H(\boldsymbol{x})} P(f(\boldsymbol{x})=-1 \mid \boldsymbol{x}) = 0,
\]可以得到:
\[
H(\boldsymbol{x})=\frac{1}{2} \ln \frac{P(f(x)=1 \mid \boldsymbol{x})}{P(f(x)=-1 \mid \boldsymbol{x})},
\]因此,有:
\[
\begin{aligned} \operatorname{sign}(H(\boldsymbol{x})) & =\operatorname{sign}\left(\frac{1}{2} \ln \frac{P(f(x)=1 \mid \boldsymbol{x})}{P(f(x)=-1 \mid \boldsymbol{x})}\right) \\ & =\left\{\begin{array}{ll}1, & P(f(x)=1 \mid \boldsymbol{x})>P(f(x)=-1 \mid \boldsymbol{x}) \\ -1, & P(f(x)=1 \mid \boldsymbol{x})<P(f(x)=-1 \mid \boldsymbol{x})\end{array}\right. \\ & =\underset{y \in\{-1,1\}}{\arg \max } P(f(x)=y \mid \boldsymbol{x}),\end{aligned}
\]
这说明,$\operatorname{sign}(H(\boldsymbol x))$达到了贝叶斯最优错误率,指数损失函数最小化则分类错误率最小化;因此,指数损失函数是分类任务原本0/1损失函数的一致的替代损失函数,应将其作为优化目标。

Adaboost在基于分布$\mathcal{D}_t$迭代生成基分类器$h_t$后,其权重$\alpha_t$应使得$\alpha_t h_t$最小化指数损失函数:\[
\begin{aligned} \ell_{\exp }\left(\alpha_{t} h_{t} \mid \mathcal{D}_{t}\right) & =\mathbb{E}_{\boldsymbol{x} \sim \mathcal{D}_{t}}\left[e^{-f(\boldsymbol{x}) \alpha_{t} h_{t}(\boldsymbol{x})}\right] \\ & =\mathbb{E}_{\boldsymbol{x} \sim \mathcal{D}_{t}}\left[e^{-\alpha_{t}} \mathbb{I}\left(f(\boldsymbol{x})=h_{t}(\boldsymbol{x})\right)+e^{\alpha_{t}} \mathbb{I}\left(f(\boldsymbol{x}) \neq h_{t}(\boldsymbol{x})\right)\right] \\ & =e^{-\alpha_{t}} P_{\boldsymbol{x} \sim \mathcal{D}_{t}}\left(f(\boldsymbol{x})=h_{t}(\boldsymbol{x})\right)+e^{\alpha_{t}} P_{\boldsymbol{x} \sim \mathcal{D}_{t}}\left(f(\boldsymbol{x}) \neq h_{t}(\boldsymbol{x})\right) \\ & =e^{-\alpha_{t}}\left(1-\epsilon_{t}\right)+e^{\alpha_{t}} \epsilon_{t} \quad \epsilon_{t}=P_{\boldsymbol{x} \sim \mathcal{D}_{t}}\left(h_{t}(\boldsymbol{x}) \neq f(\boldsymbol{x})\right)\end{aligned}
\]
上式对权重求偏导并置零可以得到:
\[
    \alpha_t = \frac{1}{2}\ln(\frac{1-\epsilon_t}{\epsilon_t})
\]

Adaboost在获得对应的分布$H_{t-1}$后,样本分布需要调整以生成新的一轮基学习器$h_t$,理想的$h_t$会纠正$H_{t-1}$的所有错误,即最小化$\ell_{exp}(H_{t-1}+h\mid \mathcal{D})$,通过泰勒展开可以得到更新公式:
\[
\begin{aligned} h_{t}(\boldsymbol{x}) & =\underset{h}{\arg \min } \ell_{\exp }\left(H_{t-1}+h \mid \mathcal{D}\right) \\ & =\underset{h}{\arg \min } \mathbb{E}_{\boldsymbol{x} \sim \mathcal{D}}\left[e^{-f(\boldsymbol{x}) H_{t-1}(\boldsymbol{x})}\left(1-f(\boldsymbol{x}) h(\boldsymbol{x})+\frac{1}{2}\right)\right] \\ & =\underset{h}{\arg \max } \mathbb{E}_{\boldsymbol{x} \sim \mathcal{D}}\left[e^{-f(\boldsymbol{x}) H_{t-1}(\boldsymbol{x})} f(\boldsymbol{x}) h(\boldsymbol{x})\right] \\ & =\underset{h}{\arg \max } \mathbb{E}_{\boldsymbol{x} \sim \mathcal{D}}\left[\frac{e^{-f(\boldsymbol{x}) H_{t-1}(\boldsymbol{x})}}{\mathbb{E}_{\boldsymbol{x} \sim \mathcal{D}}\left[e^{-f(\boldsymbol{x}) H_{t-1}(\boldsymbol{x})}\right]} f(\boldsymbol{x}) h(\boldsymbol{x})\right],\end{aligned}
\]令$\mathcal{D_t}$表示分布:\[
\mathcal{D}_{t}(\boldsymbol{x})=\frac{\mathcal{D}(\boldsymbol{x}) e^{-f(\boldsymbol{x}) H_{t-1}(\boldsymbol{x})}}{\mathbb{E}_{\boldsymbol{x} \sim \mathcal{D}}\left[e^{-f(\boldsymbol{x}) H_{t-1}(\boldsymbol{x})}\right]},
\]可以简化上式为\[
h_{t}(\boldsymbol{x})   =\underset{h}{\arg \max }\  \mathbb{E}_{\boldsymbol{x} \sim \mathcal{D}_{t}}[f(\boldsymbol{x}) h(\boldsymbol{x})] .
\]又由$f(x), h(x)\in{-1,+1}$可知:\[
\begin{aligned}
f(\boldsymbol{x}) h(\boldsymbol{x}) & =1-2 \mathbb{I}(f(\boldsymbol{x}) \neq h(\boldsymbol{x})) \\
h_{t}(\boldsymbol{x})&=\underset{h}{\arg \min } \mathbb{E}_{\boldsymbol{x} \sim \mathcal{D}_{t}}[\mathbb{I}(f(\boldsymbol{x}) \neq h(\boldsymbol{x}))]
\end{aligned}
\]
最终样本分布的更新公式为:
\[
\begin{aligned} \mathcal{D}_{t+1}(\boldsymbol{x}) & =\frac{\mathcal{D}(\boldsymbol{x}) e^{-f(\boldsymbol{x}) H_{t}(\boldsymbol{x})}}{\mathbb{E}_{\boldsymbol{x} \sim \mathcal{D}}\left[e^{\left.-f(\boldsymbol{x}) H_{t}(\boldsymbol{x})\right]}\right.} \\ & =\frac{\mathcal{D}(\boldsymbol{x}) e^{-f(\boldsymbol{x}) H_{t-1}(\boldsymbol{x})} e^{-f(\boldsymbol{x}) \alpha_{t} h_{t}(\boldsymbol{x})}}{\mathbb{E}_{\boldsymbol{x} \sim \mathcal{D}}\left[e^{\left.-f(\boldsymbol{x}) H_{t}(\boldsymbol{x})\right]}\right.} \\ & =\mathcal{D}_{t}(\boldsymbol{x}) \cdot e^{-f(\boldsymbol{x}) \alpha_{t} h_{t}(\boldsymbol{x})} \frac{\mathbb{E}_{\boldsymbol{x} \sim \mathcal{D}}\left[e^{\left.-f(\boldsymbol{x}) H_{t-1}(\boldsymbol{x})\right]}\right.}{\mathbb{E}_{\boldsymbol{x} \sim \mathcal{D}}\left[e^{\left.-f(\boldsymbol{x}) H_{t}(\boldsymbol{x})\right]}\right.}\end{aligned}
\]
}
\marginpar{\footnotesize 如果想搞懂上面一串公式究竟在推什么,强烈建议阅读南瓜书第八章(8.5——8.19)。}

Adaboost需要基学习器学习特定的数据分布,这可以通过\textbf{重赋权法}(每次对每个训练样本重新赋予权重)或者\textbf{重采样法}(每次对整个训练集重新采样后用于训练)实现。若使用重采样法,还需要\textbf{重启动机制}防止产生不合格学习器导致算法提前终止,性能下降,

\section{Bagging与随机森林}\label{sec:8.3}
并行化的集成学习可以通过在训练数据集上采样不同的样本子集,再从每个子集训练得到一个基学习器。

\textbf{Bagging}算法给定包括$m$个样本的数据集,有放回的随机采样得到T个含m个样本(可能有重复)的采样集,训练得到基学习器并结合。结合时,对分类任务采用简单投票法,回归任务采用简单平均法。

Bagging的优势在于:时间复杂度很低,可使用包外估计的方式验证泛化性能,能够降低方差,在不剪枝的决策树、神经网络等易受样本影响的学习器上效果更好。

\textbf{随机森林}是bagging的一个扩展,是基于决策树的Bagging集成的一个变种,在决策树的训练过程中引入了随机属性选择。对于基决策树的每个节点,先从该节点的属性结合中随机选择一个包括k个属性的子集,再从这个子集中选择一个最优属性进行划分。
\marginpar{\footnotesize 传统决策树是从所有的属性中选择一个最优属性。}

\section{结合策略}\label{sec:8.4}
学习器的组合可以带来三方面的好处:
\begin{itemize}
    \item 统计方面:结合多个学习器可以减小“多个假设在训练集上性能一致”的风险;
    \item 计算方面:结合多个学习器可以降低局部极小带来的糟糕泛化性能;
    \item 表示方面:结合多个学习器可以提升相应假设空间的大小。
\end{itemize}
\begin{figure}[!htbp]
	\centering
		\sidecaption{学习器结合的好处\label{fig:8.1}}{\includegraphics[width=.8\textwidth]{images/combination.png}
  }
\end{figure}
一些常见的结合策略:
\begin{enumerate}
    \item 平均法:
    常用于数值型输出,包括:
    \begin{enumerate}
        \item 简单平均法:\[H(x) = \frac{1}{T}\sum_{i=1}^T h_i(x)\]
        
        \item 加权平均法:\[
H(\boldsymbol{x})=\sum_{i=1}^{T} w_{i} h_{i}(\boldsymbol{x}), \quad w_{i} \geq 0 
,\quad \sum_{i=1}^{T} w_{i}=1
\]
    \end{enumerate}
    \item 投票法:常用于分类任务,这里的$h_i^j$代表学习器$h_i$在类别$c_j$上的输出:
    \begin{enumerate}
        \item 绝对多数投票法:\[
H(\boldsymbol{x})=\left\{\begin{array}{ll}c_{j} & \text { if } \sum_{i=1}^{T} h_{i}^{j}(\boldsymbol{x})>\frac{1}{2} \sum_{k=1}^{l} \sum_{i=1}^{T} h_{i}^{k}(\boldsymbol{x}) \\ \text { rejection } & \text { otherwise } .\end{array}\right.
\]
        \item 相对多数投票法:\[H(\boldsymbol x) = C_{\underset{j}{\arg \max }\sum_{i=1}^{T}h_i^j(x)}\]即预测为得票最多的标记。
        
        \item 加权投票法:\[H(\boldsymbol x) = C_{\underset{j}{\arg \max}\sum_{i=1}^{T} w_i h_i^j(x)}, \quad w_{i} \geq 0 
,\quad \sum_{i=1}^{T} w_{i}=1\]
    \end{enumerate}
    \item 学习法:使用另一个学习器进行结合,\textbf{Stacking}为典型代表。
    Stacking的优点在于强大的性能,缺点在于次级学习器的输入属性表示和次级学习算法对 Stacking 集成的泛化性能有很大影响。
\end{enumerate}

\section{多样性}\label{sec:8.5}
对于集成的泛化误差而言,我们有公式:
\[
E = \bar{E} - \bar{A},
\]
其中$E$代表集成泛化误差,$\hat{E}$代表所有个体学习器泛化误差的加权均值,$\hat{A}$代表所有个体学习器的加权分歧值。从该式可以得到:个体学习器准确性越高,多样性越大,其集成越好。

常见的增强个体学习器的多样性的方法包括:
\begin{itemize}
    \item \textbf{数据样本扰动}:给定初始数据集,产出不同的数据子集用于产生基学习器,一般是基于采样法,对“不稳定基学习器”(决策树,神经网络)很有效。
    \item \textbf{输入属性扰动}:选择不同的“子空间”(属性子集)训练基学习器,通过减少属性提升训练效率。
    \item \textbf{输出表示扰动}:操纵输出表示以增强多样性,例如:翻转法改变输入样本的标记;输出调剂法将分类输出改为回归输出;ECOC法将多类任务分解为一系列两类任务。
    \item \textbf{算法参数扰动}:随机设置不同的参数,例如用“负相关法”强制个体神经网络使用不同参数。
\end{itemize}

\section{本章往年考试题目}\label{sec:8.6}

本章暂无往年考试题目。

%\part{聚类}
\chapter{聚类}

\section{聚类任务}\label{sec:9.1}
\textbf{聚类}是一种典型的\textbf{无监督学习},通过对无标记的数据集的学习将数据样本划分为若干个通常不相交的“簇”。

聚类可以揭示数据的内在性质与规律,也作为分类学习任务中提取特征,判断类别的重要支撑。

\section{性能度量}\label{sec:9.2}

聚类的性能度量也被称之为“有效性指标”,用于评估聚类结果的好坏。

一个好的聚类结果,要求其“簇内相似度”高而“簇间相似度”低。

聚类性能度量大致有两类:
\begin{itemize}
    \item \textbf{外部指标}:将聚类结果与某个参考模型进行比较,例如Jaccard系数或FM指数。
    \item \textbf{内部指标}:对聚类结果直接进行考察,例如DB指数或Dunn指数。
\end{itemize}

\section{距离计算}\label{sec:9.3}
聚类的本质是划分,分组来自于合理的度量,而度量来自于距离,因此距离是聚类重要的本质。

一个距离度量需要满足的基本性质包括:
\begin{itemize}
    \item 非负性:$\operatorname{dist}(\boldsymbol x_i,\boldsymbol x_j) \ge 0$;
    \item 同一性:$\operatorname{dist}(\boldsymbol x_i, \boldsymbol x_j) = 0, \; \operatorname{iff} \;\boldsymbol x_i = \boldsymbol x_j$
    \item 对称性:$\operatorname{dist}(\boldsymbol x_i, \boldsymbol x_j) = \operatorname{dist}(\boldsymbol x_j, \boldsymbol x_i)$
    \item 直递性:$\operatorname{dist}(\boldsymbol x_i, \boldsymbol x_j) \le \operatorname{dist}(\boldsymbol x_i, \boldsymbol x_k) + \operatorname{dist}(\boldsymbol x_k, \boldsymbol x_j)$
\end{itemize}

给定样本$\boldsymbol x_i = (x_{i 1};x_{i 2};...;x_{in})$与$\boldsymbol x_j = (x_{j 1};x_{j 2};...;x_{jn})$,若样本属性有序,常用的距离度量为\textbf{闵可夫斯基距离}:\[
\operatorname{dist}_{m k}\left(x_{i}, x_{j}\right)=\left(\sum_{u=1}^{n}\left|x_{i u}-x_{j u}\right|^{p}\right)^{\frac{1}{p}}
\]

当$P=1$时,闵可夫斯基距离即为曼哈顿距离;$P=2$时即为欧式距离。

当样本属性无序时,可使用\textbf{VDM},令$m_{u,a}$表示属性$u$上取值为$a$的样本数,$m_{u,a,i}$表示在第$i$个样本簇上属性$u$上取值为$a$的样本数,$k$为样本簇数,则属性$u$上两个离散值a与b之间的VDM距离为:
\[
V D M_{p}(a, b)=\sum_{i=1}^{k}\left|\frac{m_{u, a, i}}{m_{u, a}}-\frac{m_{u, b, i}}{m_{u, b}}\right|^{p}
\]

将闵可夫斯基距离与VDM结合使用可以处理混合属性,假设有$n_c$个有序属性,$n-n_c$个无序属性,令有序属性在无序属性之前,则有\[
\operatorname{MinkovDM}_{p}\left(\boldsymbol{x}_{i}, \boldsymbol{x}_{j}\right)=\left(\sum_{u=1}^{n_{c}}\left|x_{i u}-x_{j u}\right|^{p}+\sum_{u=n_{c}+1}^{n} \operatorname{VDM}_{p}\left(x_{i u}, x_{j u}\right)\right)^{\frac{1}{p}}
\]

\section{原型聚类}\label{sec:9.4}
原型聚类,也被称为“基于原型的聚类”,其假设聚类结构能够\textbf{通过一组原型刻画}。

\subsection{K-means聚类}\label{sec:9.4.1}
K-均值算法以均值向量$\boldsymbol \mu_i = \frac{1}{|C_i|}\sum_{\boldsymbol x\in C_i}\boldsymbol x$为原型,其算法大致流程为:
\begin{enumerate}
    \item 随机选取k个样本点作为簇中心;
    \item 将其他样本点根据其与簇中心的距离,划分给最近的簇;
    \item 更新各簇的均值向量,将其作为新的簇中心;
    \item 若簇中心未发生更改则停止,否则返回第2步。
\end{enumerate}

\subsection{学习向量量化LVQ}
学习向量量化假设数据样本带有类别标记,通过学习的方式尝试找到一组原型向量刻画聚类结构。

通过聚类来形成类别的“子类”结构;每个聚类对应于类别的一个子类

\subsection{高斯混合聚类}
高斯混合聚类采用概率模型来表达聚类原型,具体而言,使用高斯概率分布:

对于n维样本空间$\mathcal{X}$中的随机向量$\boldsymbol x$,若其服从高斯分布,则其概率密度函数为:\[
p(\boldsymbol{x})=\frac{1}{(2 \pi)^{\frac{n}{2}}|\boldsymbol{\Sigma}|^{\frac{1}{2}}} e^{-\frac{1}{2}(\boldsymbol{x}-\boldsymbol{\mu})^{\mathrm{T}} \boldsymbol{\Sigma}^{-1}(\boldsymbol{x}-\boldsymbol{\mu})}
\]
基于此可以定义高斯混合分布:
\[
p_{\mathcal{M}}(\boldsymbol{x})=\sum_{i=1}^{k} \alpha_{i} \cdot p\left(\boldsymbol{x} \mid \boldsymbol{\mu}_{i}, \boldsymbol{\Sigma}_{i}\right)
\]

在这种情况下,样本的生成过程为:首先根据$\alpha_1, \alpha_2, ...,\alpha_k$定义的先验分布选择高斯混合成分,其中$\alpha_i$为选择第i个混合成分的概率;然后根据被选择的混合成分的概率密度函数进行采样, 从而生成对应的样本。



\section{密度聚类}\label{sec:9.5}

密度聚类,也被称为“基于密度的聚类”,其假设聚类结构能够通过样本分布之间的\textbf{紧密程度}确定。这类方法从样本密度的角度来考察样本之间的可连接性,并基于可连接样本不断扩展聚类簇。

代表算法:DBSCAN, OPTICS, DENCLUE

\subsection{DBSCAN}
DBSCAN基于一组“邻域”参数$(\epsilon, MinPts)$刻画样本分布的紧密程度,其核心思想在于通过邻域建立样本间的可达路径并形成联通分支(等价类)。
其关键概念为:
\begin{itemize}
    \item 核心对象(core object):若$\boldsymbol x_j$的 $\epsilon$-邻域至少包含 $MinPts$ 个样本,形式化表达为$|N_\epsilon(\boldsymbol x_j)| \ge MinPts$,则$\boldsymbol x_j$是一个核心对象。
    \item 密度直达(directly density-reachable):若$\boldsymbol x_j$位于$\boldsymbol x_i$的$\epsilon$-邻域中,且$\boldsymbol x_i$是核心对象,则称$\boldsymbol x_j$由$\boldsymbol x_i$密度直达;
    \item 密度可达(density-reachable):对$\boldsymbol x_i$与$\boldsymbol x_j$,若存在样本序列$\boldsymbol p_1, \boldsymbol p_2,...,\boldsymbol p_n$,其中$\boldsymbol p_1 = \boldsymbol x_i, \boldsymbol p_n = \boldsymbol x_j$且 $p_{i+1}$ 由 $p_i$ 密度直达, 则称 $\boldsymbol x_j$由 $\boldsymbol x_i$密度可达。
    \item 密度相连(density-connected):对 $\boldsymbol x_i$与$\boldsymbol x_j$, 若存在 $\boldsymbol x_k$使得$\boldsymbol x_i$与$\boldsymbol x_j$均由$x_k$密度可达,则称$\boldsymbol x_i$与$\boldsymbol x_j$密度相连。
\end{itemize}

由此,DBSCAN将“簇”定义为:由密度可达关系导出的最大密度相连样本集合。

\section{层次聚类}\label{sec:9.6}
层次聚类假设数据集能够产生不同粒度的聚类结构,这类方法试图在不同层次上对数据集进行划分,形成树形的聚类结构。

代表算法:AGNES(自底向上),DIANA(自顶向下)。

\subsection{AGNES}
AGNES首先将数据集中的每个样本看作一个初始聚类簇,然后不断找出聚类最近的两个聚类簇进行合并,直至所有的数据均属于一个簇为止。通过对生成的树状图在不同尺度上进行分割,可以得到相应的簇划分结果。

\section{本章往年考试题目}\label{sec:9.7}

\ex{2022年考试原题}{ex_ref}{阐述k均值聚类的思想;在算法中,随机选择k个簇中心这一步怎么优化?簇中心的个数k应当怎么选?}

K均值聚类的思想详见\ref{sec:9.4.1}。

K均值聚类的优化:
\begin{itemize}
    \item \textbf{K-Means++}:首先随机选择一个数据点作为第一个初始聚类中心,然后对于每个未被选择的数据点,计算其与已有聚类中心之间的最小距离,并根据该距离的概率分布选择下一个聚类中心。通过这种方式,K-means++算法能够使得初始聚类中心之间距离较远,从而避免陷入局部最优解。
    \item \textbf{基于密度的初始化}:基于密度的初始化方法考虑数据点的分布密度,选择密度较高的区域作为初始聚类中心。这种方法能够更好地反映数据的内在结构,使得聚类结果更加合理。一种常见的基于密度的初始化方法是选择局部密度峰值作为初始聚类中心。
\end{itemize}

K的选择:
\begin{itemize}
    \item \textbf{轮廓系数}:轮廓系数是一种评估聚类效果的指标,它综合考虑了同一聚类内数据点的紧凑度和不同聚类间数据点的分离度。通过计算不同K值下的轮廓系数,可以选择使得轮廓系数最大的K值作为最优聚类数。
    \item \textbf{肘部法则}:肘部法则通过观察聚类误差平方和(SSE)随K值变化的曲线来确定最优聚类数。当K值较小时,增加K值会显著降低SSE;而当K值达到某个阈值后再增加K值对SSE的降低效果不再明显。这个阈值对应的K值即为最优聚类数。
\end{itemize}


\ex{2018年考试原题·附加题}{ex_ref}{写出距离度量的四个性质,并证明闵可夫斯基距离满足这些性质。}
\ex{2020年考试原题}{ex_ref}{写出闵可夫斯基距离的表达式,并阐述距离度量需要满足的条件,给出欧式距离与闵可夫斯基距离的关联。}
非负性与对称性由绝对值可知显然。

同一性$x=y \Rightarrow d(x, y)=0$显然,反向则有:\[
\sum_{i=1}^{n}\left|x_{i}-y_{i}\right|^{p}=0 \Rightarrow \forall i,\left|x_{i}-y_{i}\right|=0 \Rightarrow x_{i}=y_{i} \Rightarrow x=y
\]

直递性证明较为复杂,其正确性由Minkowski不等式保证,本身是从Hölder不等式推出的,此处忽略。

欧式距离 = 闵可夫距离取P等于2的特值。

\ex{2023/2025年考试原题}{ex_ref}{写出密度聚类和层次聚类的代表性算法及步骤,并分别说明其关键假设。}
见\ref{sec:9.5}与\ref{sec:9.6}节。

%\part{降维与度量学习}
\chapter{降维与度量学习}

\section{k-近邻学习}\label{sec:10.1}
kNN是一种常用的监督学习方式,其主要思想是基于某种距离度量找到训练集中与其最接近的k个训练样本,基于这k个邻居的信息进行预测。

该方法是\textbf{“懒惰学习”}的典型代表:没有显式的训练过程,仅仅把训练样本收集起来,待收到测试样本之后再进行处理。

“最近邻分类器”的基本性质是:给定测试样本$\boldsymbol x$与最近邻样本$\boldsymbol z$,最近邻分类器出错的概率是$\boldsymbol x,\boldsymbol z$类别标记不同的概率:$P(err) = 1-\sum_{c\in \mathcal{Y}}P(c\mid \boldsymbol x)P(c\mid \boldsymbol z)$。
假设样本独立同分布且对任意测试样本总能在任意近的范围内找到训练样本$\boldsymbol z$,则贝叶斯最优分类器结果$c* = \arg\max_{c\in \mathcal{Y}}P(c\mid \boldsymbol x)$的出错概率为:
\[
\begin{aligned} P(e r r) & =1-\sum_{c \in \mathcal{Y}} P(c \mid \boldsymbol{x}) P(c \mid \boldsymbol{z}) \\ & \simeq 1-\sum_{c \in \mathcal{Y}} P^{2}(c \mid \boldsymbol{x}) \\ & \leqslant 1-P^{2}\left(c^{*} \mid \boldsymbol{x}\right) \\ & =\left(1+P\left(c^{*} \mid \boldsymbol{x}\right)\right)\left(1-P\left(c^{*} \mid \boldsymbol{x}\right)\right) \\ & \leqslant 2 \times\left(1-P\left(c^{*} \mid \boldsymbol{x}\right)\right)\end{aligned}
\]

即最近邻分类器的泛化错误率不会超过贝叶斯最优分类器的两倍。

\section{低维嵌入}\label{sec:10.2}

上述讨论基于密采样假设:样本的每个$\epsilon$-邻域都有近邻;但是这个假设在实际情况中可能发生\textbf{维度灾难}。
\marginnote{\footnotesize 维度增加带来的样本数增加是指数级的。}
并且,高维空间计算距离度量难度大,近邻容易不准。

为解决这个方法,我们选择进行降维,找到与学习任务密切相关的某个高维空间的\textbf{低维“嵌入”}。

在低维空间中保持原始空间中样本之间的距离的方法是\textbf{多维缩放}。其主要思路为\textbf{“内积保距”},即寻找一个低维子空间,使得距离与样本原有距离近似不变。

多维缩放方法的技巧为特征值分解,即分解内积矩阵的特征值,取一部分最大特征值作为低维空间。由谱分布长尾可知,存在相当数量的小特征值,删除对距离影响不大。

\section{流形学习}\label{sec:10.3}
流形学习是一类借鉴了拓扑流形概念的降维方法,拓扑流形在局部具有欧氏空间的性质,能用欧氏距离进行距离计算,这使得将低维流形嵌入高维空间中,就可以在局部建立降维映射关系。

\subsection{等度量映射ISOMAP}
ISOMAP的关键思路为计算“测地线距离”。对每个点基于欧式距离找出其近邻点,基于最短路径算法近似任意两点之间的测地线距离,在得到任意两点的距离之后通过MDS(多维缩放)获得低维嵌入。

\subsection{局部线性嵌入LLE}
LLE的关键思路为重构权值,试图保持邻域内样本之间的线性关系。LLE为每个样本构造近邻下标集合$Q_i$,再计算出基于$Q_i$的线性重构系数$\boldsymbol w_i$:
\[
\min _{w_{1}, \ldots, w_{m}} \sum_{i=1}^{m}\left\|x_{i}-\sum_{j \in Q_{i}} w_{i j} x_{j}\right\|_{2}^{2} \quad \text{s.t.}
\quad \sum_{j \in Q_{i}} w_{i j}=1,
\]LLE若要在低维空间中保持$w_{i j}$不变,则可求解下式:\[
\min _{z_{1}, \ldots, z_{m}} \sum_{i=1}^{m}\left\|z_{i}-\sum_{j \in Q_{i}} w_{i j} z_{j}\right\|_{2}^{2}
\]

\section{度量学习}\label{sec:10.4}
机器学习降维数据的主要目的是寻找合适的低维空间,而每个空间对应了在样本属性上定义的一个距离度量,而\textbf{度量学习}的动机即为直接“学习”一个合适的距离度量。

\thm{马氏距离}{thm_ref}{

对于两个$d$维样本$\boldsymbol x_i, \boldsymbol x_j$,他们之间的平方欧氏距离可写为:\[
\operatorname{dist}_{\mathrm{ed}}^{2}\left(\boldsymbol{x}_{i}, \boldsymbol{x}_{j}\right)=\left\|\boldsymbol{x}_{i}-\boldsymbol{x}_{j}\right\|_{2}^{2}=\operatorname{dist}_{i j, 1}^{2}+\operatorname{dist}_{i j, 2}^{2}+\ldots+\operatorname{dist}_{i j, d}^{2}\ ,
\]
其中$\operatorname{dist}_{i j, d}$代表$\boldsymbol x_i, \boldsymbol x_j$在第$k$维上的距离,对每个属性引入一个属性权重$\boldsymbol w$代表属性的重要度,可得:\[
\begin{aligned}
\operatorname{dist}_{\mathrm{ed}}^{2}\left(\boldsymbol{x}_{i}, \boldsymbol{x}_{j}\right) & =\left\|\boldsymbol{x}_{i}-\boldsymbol{x}_{j}\right\|_{2}^{2}=w_1\cdot\operatorname{dist}_{i j, 1}^{2}+w_2\cdot\operatorname{dist}_{i j, 2}^{2}+\ldots+w_d\cdot\operatorname{dist}_{i j, d}^{2} \\
& = (\boldsymbol{x}_{i}-\boldsymbol{x}_{j})^\mathrm{T}\mathbf{W}(\boldsymbol{x}_{i}-\boldsymbol{x}_{j})
\end{aligned}
\]

其中$\mathbf{W}$是一个对角矩阵,对角上每个元素均为权值$w_i$,但对角矩阵代表属性之间彼此无关,而属性有时正相关,因此将$\mathbf{W}$替换为一个普通的半正定对称矩阵$\mathbf{M}$,即得到了马氏距离:\[
\operatorname{dist}_{\mathrm{mah}}^{2}\left(\boldsymbol{x}_{i}, \boldsymbol{x}_{j}\right)=\left(\boldsymbol{x}_{i}-\boldsymbol{x}_{j}\right)^{\mathrm{T}} \mathbf{M}\left(\boldsymbol{x}_{i}-\boldsymbol{x}_{j}\right)=\left\|\boldsymbol{x}_{i}-\boldsymbol{x}_{j}\right\|_{\mathbf{M}}^{2}
\]
其中$\mathbf{M}$亦称“度量矩阵”,度量学习的目标就是学习$\mathbf{M}$。

欧氏距离的缺点是将各个方向视作同等重要,而马氏距离则能反映表面欧氏距离所不能反映的特征之间的相关性与特征尺度。
}

对$\mathbf{M}$进行学习需要设置目标。

目标其一为\textbf{结合具体分类器的性能},例如,以近邻分类器的性能为目标,即可得到经典的NCA算法;目标其二是\textbf{引入领域知识},例如,可以通过建立“必连”(样本相似)与“勿连”(样本不相似)约束集合获得度量矩阵。

\section{PCA 主成分分析} \label{sec:10.5}
对于正交属性空间中的样本点,如果用一个超平面对所有样本进行恰当表达,则这个超平面应当具有如下的性质:
\begin{itemize}
    \item 最近重构性:样本点到这个超平面距离都足够近;
    \item 最大可分性:样本点在这个超平面上的投影尽可能地分开。
\end{itemize}

可以根据这两种不同的性质,给出PCA的两种等价推导。

\thm{最近重构性推导}{thm_ref}{假设数据样本进行了中心化($\sum_i\boldsymbol x_i = 0$),且投影变换后得到的新坐标系为$\{\boldsymbol w_1,\boldsymbol w_2, \ldots,\boldsymbol w_d\}$;
则通过丢弃一部分坐标将维度降低至$d'\lt d$,得到样本点$\boldsymbol x_i$在低维坐标系中的投影$\boldsymbol z_i = (z_{i1};z_{i2};\ldots;z_{id'}), \  z_{ij} = \boldsymbol w^\mathrm{T}_j\boldsymbol x_i,$基于$\boldsymbol z_i$来投影重构$\boldsymbol x_i$,可以得到$\hat{\boldsymbol x}_i = \sum_{j=1}^{d'}z_{ij}\boldsymbol w_j$。
于是可以得到原样本点与基于投影重构的样本点之间的距离为:\[
\begin{aligned} \sum_{i=1}^{m}\left\|\sum_{j=1}^{d^{\prime}} z_{i j} \boldsymbol{w}_{j}-\boldsymbol{x}_{i}\right\|_{2}^{2} & =\sum_{i=1}^{m} \boldsymbol{z}_{i}^{\mathrm{T}} \boldsymbol{z}_{i}-2 \sum_{i=1}^{m} \boldsymbol{z}_{i}^{\mathrm{T}} \mathbf{W}^{\mathrm{T}} \boldsymbol{x}_{i}+\text { const } \\ & \propto-\operatorname{tr}\left(\mathbf{W}^{\mathrm{T}}\left(\sum_{i=1}^{m} \boldsymbol{x}_{i} \boldsymbol{x}_{i}^{\mathrm{T}}\right) \mathbf{W}\right) .\end{aligned}
\]
$\boldsymbol w_j$为正交基,$\sum_i \boldsymbol x_i\boldsymbol x_i^T$是协方差矩阵,于是由最近重构性有:
\[
\min _{\mathbf{W}}-\operatorname{tr}\left(\mathbf{W}^{\mathrm{T}} \mathbf{X} \mathbf{X}^{\mathrm{T}} \mathbf{W}\right) \quad \text { s.t. } \mathbf{W}^{\mathrm{T}} \mathbf{W}=\mathbf{I} 
\]
这是主成分分析的优化目标。}

\thm{最大可分性推导}{thm_ref}{样本点在新空间中超平面上的投影为$\mathbf{W}^\mathrm{T}\boldsymbol x_i$,若所有样本点的投影尽可能分开,则应该使得投影后样本点的方差最大化。

投影后样本的协方差矩阵为$\sum_i\mathbf{W}^\mathrm{T}\boldsymbol x_i\boldsymbol x_i^\mathrm{T}\mathbf{W}$,故优化目标为:\[
\max _{\mathbf{W}} \ \operatorname{tr}\left(\mathbf{W}^{\mathrm{T}} \mathbf{X} \mathbf{X}^{\mathrm{T}} \mathbf{W}\right) \quad \text { s.t. } \mathbf{W}^{\mathrm{T}} \mathbf{W}=\mathbf{I} 
\]
不难看出,最大可分性推导出的优化目标与最近重构性的等价。
}

对求解得到的优化目标使用拉格朗日乘子法可得\[
\mathbf{X} \mathbf{X}^{\mathrm{T}} \mathbf{W}=\lambda \mathbf{W}
\]

此时对$\mathbf{X} \mathbf{X}^{\mathrm{T}}$做特征值分解,并对求得的特征值排序,取前$d'$大的特征值对应的特征向量构成$\mathbf{W} = (\boldsymbol w_1, \boldsymbol w_2,\ldots, \boldsymbol w_{d'})$,这就是主成分分析的解。$d'$的选取可以是用户指定、交叉验证的结果或者设置重构阈值得到的结果。

\textbf{PCA是最常用的降维方法}。例如,在人脸识别中,该技术被称之为\textbf{特征脸}。

\section{本章往年考试题目}\label{sec:10.6}

\ex{2018,19,20,21,23,25年考试原题}{ex_ref}{PCA的最近重构性与最大可分性是如何推导得到优化目标的?}
见\ref{sec:10.5}节。

\ex{2019年考试原题}{ex_ref}{特征脸是什么,特征脸对降维有什么帮助?}
特征脸是主成分分析在人脸识别中的别称。

特征脸在降维的过程中能够增大样本的采样密度,且一定程度上起到降噪的作用。

\ex{2019年考试原题}{ex_ref}{马氏距离表达式,M矩阵的要求,马氏距离和欧式距离的关系?}
见\ref{sec:10.4}节。

\ex{2022年考试原题}{ex_ref}{维度灾难是什么?ISOMAP、LLE的过程和思想?}
维度灾难:高维空间给距离的计算带来很大的麻烦;当维数很高时,甚至连计算内积都不再容易;更严重的是,样本变得稀疏,近邻容易不准。

ISOMAP、LLE的详细内容请见\ref{sec:10.3}节。

%\part{特征选择与稀疏表示}
\chapter{特征选择与稀疏表示}

\section{子集搜索与评价}\label{sec:11.1}
对于一个学习任务,给定属性集(我们将属性称为“特征”),我们将对当前学习任务有用的属性称为\textbf{相关特征},无用的属性称为\textbf{无关特征},通过其他特征可以推演出来的特征称为\textbf{冗余特征}。

从给定的特征集合中选择出相关特征子集,而不丢弃重要特征的过程称为\textbf{特征选择}。进行特征选择的原因在于,其一、使用部分特征能有效减轻维数灾难问题;其二、去除不相关属性能降低学习任务难度。

在特征选择的过程中,涉及到两个关键环节:如何根据选择的特征子集的评价结果,获取下一个特征子集;如何评价选择的特征子集的好坏。

\textbf{子集搜索}解决的是第一个环节的问题,其策略包括:
\begin{itemize}
    \item \textbf{前向搜索}:将每个特征看作一个候选子集,不断将单个特征加入选择集合,直至效果不如增加前。
    \item \textbf{后向搜索}:从完整的特征集合开始,每次尝试去掉一个无关特征,直至无法再去除。
    \item \textbf{双向搜索}:每一轮增加选定相关特征并去除无关特征。
\end{itemize}
\marginnote{\footnotesize 不难发现上述策略都是贪心的。}

\textbf{子集评价}解决的是第二个环节的问题,我们通过计算每一个特征子集的信息增益作为子集的评价准则,\[
\begin{aligned}
    \operatorname{Gain}(A) & =\operatorname{Ent}(D)-\sum_{v=1}^{V} \frac{\left|D^{v}\right|}{|D|} \operatorname{Ent}\left(D^{v}\right) \\
    \operatorname{Ent}(D) & = -\sum_{i=1}^{|\mathcal{Y}|} p_k\log_2p_k
\end{aligned}
\]信息增益越大,特征子集包含的有助于分类的信息越多。

\section{过滤式选择}\label{sec:11.2}
过滤式选择方法先\textbf{对数据集进行特征选择},再训练学习器,特征选择过程与后续学习器无关。
这种方法的经典代表为Relief算法,通过设计\textbf{“相关统计量”}来度量特征的重要性。

对于Relief算法而言,给定训练集$\{(\boldsymbol x_1, y_1), (\boldsymbol x_2, y_2), \ldots,(\boldsymbol x_m, y_m) \}$,对每个示例$\boldsymbol x_i$,算法找到其同类样本中的最近邻$\boldsymbol x_{i,nh}$与异类样本中的最近邻$\boldsymbol x_{i,nm}$作为猜中近邻near-hit与猜错近邻near-miss,并计算相关统计量针对属性$j$的分量:
\[
\delta^{j}=\sum_{i}-\operatorname{diff}\left(x_{i}^{j}, x_{i, \mathrm{nh}}^{j}\right)^{2}+\operatorname{diff}\left(x_{i}^{j}, x_{i, \mathrm{~nm}}^{j}\right)^{2}
\]
其中diff表示两者之间的差距(距离),$x_a^j$代表样本$\boldsymbol x_a$在属性$j$上的取值。

\marginnote{\footnotesize Relief主要用于二分类问题,运行效率很高,其变体Relief-F可以处理多分类问题,本笔记略过。}
从上述描述可知,若猜中近邻与$\boldsymbol x_i$的距离小于猜错近邻与$\boldsymbol x_i$的距离,则说明该属性对于区分同类异类是有益的,应增大该统计量分量;否则则减小。最终对基于不同样本得到的估计结果做平均,分量值越大,该属性分类能力越强。

\section{包裹式选择}\label{sec:11.3}

包裹式选择直接将\textbf{最终使用的学习器性能}作为特征子集的评价准则,其目的是为了给学习器选择最有利于其性能发挥的特征子集。从最终学习器性能来看,包裹式强于过滤式。但由于需要多次训练学习器,计算开销一般大得多。

这类选择法的代表为LVW算法。其通过在循环的每一轮随机产生一个特征子集,通过交叉验证推断当前特征子集的误差,并不断循环,在多个随机产生的特征子集中选择误差最小的特征子集作为最终解。

\section{嵌入式选择与L1正则化}\label{sec:11.4}
嵌入式特征选择是将\textbf{特征选择过程与学习器训练过程融为一体},两者在同一个优化过程中完成。

给定数据集$D=\{ (\boldsymbol x_1, y_1),(\boldsymbol x_2, y_2), \ldots,(\boldsymbol x_m, y_m)\}$,考虑最简单的线性回归模型,以平方误差为损失函数,则优化目标为:
\[
\min _{\boldsymbol{w}} \sum_{i=1}^{m}\left(y_{i}-\boldsymbol{w}^{\top} \boldsymbol{x}_{i}\right)^{2}
\]
特征较多而样本数相对较少时,容易陷入过拟合,此时可以引入正则化项:
\[
\begin{aligned}
引入L2正则项 & :\min _{\boldsymbol{w}} \sum_{i=1}^{m}\left(y_{i}-\boldsymbol{w}^{\top} \boldsymbol{x}_{i}\right)^{2}+\lambda\|\boldsymbol{w}\|_{2}^{2} \\
引入L1正则项 & :\min _{\boldsymbol{w}} \sum_{i=1}^{m}\left(y_{i}-\boldsymbol{w}^{\top} \boldsymbol{x}_{i}\right)^{2}+\lambda\|\boldsymbol{w}\|_{1}
\end{aligned}
\]

前者被称之为\textbf{岭回归},后者被称之为\textbf{LASSO}。

使用L1范数正则化的好处是,更容易获得稀疏解,求得的$\boldsymbol w$会有更少的非零分量。

\begin{figure}[!htbp]
	\centering
		\sidecaption{假设$\boldsymbol x$仅有两个属性,基于这两个属性做坐标轴画出的范数等值线。\label{fig:11.1}}{\includegraphics[width=.8\textwidth]{images/L1.png}}
\end{figure}

引入正则化项之后,解要在平方误差项与正则化项中做折中,即出现在平方误差项等值线与正则化项等值线相交处。从图\ref{fig:11.1}中可以看出,$L_1$范数时常出现在坐标轴上,可以产生$w_1$或者$w_2$为0的稀疏解。

基于L1正则化的结果是得到了仅采用一部分初始特征的模型,可以认为L1正则化学习就是一种嵌入式特征选择方法。

\section{稀疏表示}\label{sec:11.5}
把数据集$D$考虑为一个矩阵,其每行对应于一个样本,每列对应一个特征,特征选择所考虑的问题是\textbf{特征具有稀疏性},可以通过特征选择去除这些列,使学习器训练过程仅需要在较小的矩阵上进行。另外一种稀疏性是,矩阵中有很多个零元素,且并非整行整列的出现,如字频矩阵(行对应文档,列对应字词,交汇处为出现频率)。

稀疏表达的优势是:文本数据使用字频表示后能够使大多数问题变得线性可分;稀疏样本能够高效存储。

基于稀疏表达的优势,给定稠密表示的数据集$D$,为这个样本找到合适的字典,将其转化为稀疏表示的这一过程称为\textbf{字典学习}。

具体而言,给定数据集$\{\boldsymbol x_1, \boldsymbol x_2,\ldots, \boldsymbol x_m\}, \boldsymbol x_i \in \mathbb{R}^{n\times k}$,字典学习最简单的优化形式为:
\[
    \min_{\mathbf{B}, \boldsymbol \alpha_i} \sum_{i=1}^{m} \|\mathrm{x}_i - \mathbf{B}\boldsymbol \alpha_i \|^2_2 + \lambda\sum_{i=1}^{m}\|\boldsymbol \alpha_i\|_1,
\]
其中$B\in\mathbb{R}^{d\times k}$为对应的字典矩阵,$\boldsymbol \alpha_i \in \mathbb{R}^k$为样本对应的稀疏表示,我们学习得到上述两个表示。k为用户自定义的字典词汇量。

矩阵补全希望通过已知的部分信息来获取全部信息,基于压缩感知的思想恢复出完整的信号,其对应形式为:

\marginnote{\footnotesize 压缩感知的思想是基于信号本身的稀疏性从部分观测样本中恢复原信号,一般分为“感知测量”与“重构恢复”两方面。矩阵补全对应的是“重构恢复”方面。}
\[
\min _{X} \operatorname{rank}(X) \quad
\quad X_{i j}=A_{i j},(i, j) \in \Omega
\]

这是一个NP难问题。

\section{本章往年考试题目}\label{sec:11.6}

\ex{2018年考试原题}{ex_ref}{子集搜索的方法是什么?子集评价的方法是什么?}
详见\ref{sec:11.1}节。

\ex{2019年考试原题}{ex_ref}{L1范数为什么可以得到稀疏解?参数$\lambda$对于稀疏解的影响是什么?}
稀疏解的解释详见\ref{sec:11.4}节。

$\lambda$是控制正则化强度的超参数,它决定了模型在拟合训练数据与参数稀疏性之间的权衡:它越大,模型越稀疏,越多属性会变为0;越小,模型越复杂,会保留越多的属性。

\ex{2019/2022/2023/2025年考试原题}{ex_ref}{特征选择的三种方法区别是什么?Relief算法属于哪种方法?LVW算法又属于哪种方法?}

区别:
\begin{itemize}
    \item 嵌入式:将特征选择过程与学习器训练过程融为一体,两者在同一个优化过程中完成,在学习器中自动进行特征选择
    \item 包裹式:将最终使用的学习器的性能作为特征子集的评价标准,为给定选择器选择最有利于其性能、“量身定做”的特征子集。
    \item 过滤式:先用特征选择过程过滤原始数据,再用过滤后的特征来训练模型;特征选择过程和后续学习器无关。
\end{itemize}
嵌入式例子:L1正则化;包裹式例子:LVW;过滤式例子:Relief。



%\part{半监督学习}
\chapter{半监督学习}

\section{未标记样本}\label{sec:12.1}
监督学习的成功依赖于大量高质量的标注样本,而实际上许多任务难以获取标注数据,因此需要同时利用有标记样本与未标记样本构建泛化性能良好的模型。

让学习器不依赖外界交互,自动地利用未标记样本来提升学习性能的方法被称为\textbf{半监督学习}。半监督学习对利用的未标记样本做出一系列假设:

\begin{itemize}
    \item 聚类假设:假设数据存在簇结构, 同一簇的样本属于同一类别;
    \item 流形假设:假设数据分布在一个流形结构上, 临近的样本具有相似的输出值。
\end{itemize}

\marginnote{\footnotesize 这两个假设的本质都是“相似的样本拥有相似的输出”。}

\section{生成式方法}\label{sec:12.2}
生成式方法顾名思义,直接基于生成式模型,假设所有数据(无论是否有标记)都由同一个潜在的模型生成,将未标注数据的标记看作潜在模型的缺失参数,通过EM算法极大似然估计求解。

\thm{EM算法求解}{thm_ref}{假设样本$\boldsymbol x$由高斯混合模型生成,每个类别$y\in\mathcal{Y}=\{1,2,\ldots,N\}$对应一个高斯混合成分,则有:\[
p(\boldsymbol{x})=\sum_{i=1}^{k} \alpha_{i} \cdot p\left(\boldsymbol{x} \mid \boldsymbol{\mu}_{i}, \boldsymbol{\Sigma}_{i}\right).
\]其中混合系数$\alpha_i \ge 0, \sum_{i=1}^{n}\alpha_i =1$,$p(\boldsymbol x|\boldsymbol \mu_i, \boldsymbol{\Sigma}_{i})$是样本属于第i个高斯混合成分$(\boldsymbol \mu_i, \boldsymbol{\Sigma}_{i})$的概率。
由最大化后验概率可知:
\[
\begin{array}{l}f(\boldsymbol{x})=\underset{j \in \mathcal{Y}}{\operatorname{argmax}} p(y=j \mid \boldsymbol{x}) \\ =\underset{j \in \mathcal{Y}}{\operatorname{argmax}} \sum_{i=1}^{k} p(y=j, \Theta=i \mid \boldsymbol{x})  \\ =\underset{j \in \mathcal{Y}}{\operatorname{argmax}} \sum_{i=1}^{k} p(y=j \mid \Theta=i, \boldsymbol{x}) \cdot p(\Theta=i \mid \boldsymbol{x})\end{array}
\]其中$ p(\Theta=i \mid x)=\frac{\alpha_{i} p\left(\boldsymbol{x} \mid \boldsymbol{\mu}_{i}, \boldsymbol{\Sigma}_{i}\right)}{\sum_{i=1}^{k} \alpha_{i} p\left(\boldsymbol{x} \mid \boldsymbol{\mu}_{i}, \boldsymbol{\Sigma}_{i}\right)} $代表样本$\boldsymbol x$由第i个高斯混合成分生成的概率(不涉及样本标记),$p(y=j\mid\Theta=i, \boldsymbol x )$为样本$\boldsymbol x$由第i个高斯混合成分生成且类别为j的概率(需知样本标记),由于每个类别对应一个高斯混合成分,可直接代换为$p(y=j\mid\Theta=i)$。

根据有标记样本集$D_l = \{(\boldsymbol x_1, y_1),(\boldsymbol x_2, y_2),\ldots,(\boldsymbol x_l, y_l)\}$与无标记数据集$D_u = \{\boldsymbol x_{l+1}, \boldsymbol x_{l+2}, \ldots,\boldsymbol x_{l+u}\}, l+u= m$,假设所有样本独立同分布且均由同一个高斯混合模型生成,即可通过对数似然估计高斯混合模型的参数$\{(\alpha_i, \boldsymbol \mu_i, \boldsymbol \Sigma_i)\mid 1 \le i \le N \}$:
\[
\begin{aligned} \ln p\left(D_{l} \cup D_{u}\right)= & \sum_{\left(\boldsymbol{x}_{j}, y_{j}\right) \in D_{l}} \ln \left(\sum_{i=1}^{k} \alpha_{i} \cdot p\left(\boldsymbol{x}_{j} \mid \boldsymbol{\mu}_{i}, \boldsymbol{\Sigma}_{i}\right) \cdot p\left(y_{j} \mid \Theta=i, \boldsymbol{x}_{j}\right)\right) \\ & +\sum_{\boldsymbol{x}_{j} \in D_{u}} \ln \left(\sum_{i=1}^{k} \alpha_{i} \cdot p\left(\boldsymbol{x}_{j} \mid \boldsymbol{\mu}_{i}, \boldsymbol{\Sigma}_{i}\right)\right) .\end{aligned}
\]

EM模型求解参数主要分为两步:

E步:根据当前模型参数计算未标记样本$x_j$属于各高斯混合成分的概率:\[
\gamma_{j i}=\frac{\alpha_{i} \cdot p\left(\boldsymbol{x}_{j} \mid \boldsymbol{\mu}_{i}, \boldsymbol{\Sigma}_{i}\right)}{\sum_{i=1}^{N} \alpha_{i} \cdot p\left(\boldsymbol{x}_{j} \mid \boldsymbol{\mu}_{i}, \boldsymbol{\Sigma}_{i})\right.}
\]

M步:基于$\gamma_{ji}$更新模型参数:\[
\boldsymbol{\mu}_{i}=\frac{1}{\sum_{\boldsymbol{x}_{j} \in D_{u}} \gamma_{j i}+l_{i}}\left(\sum_{\boldsymbol{x}_{j} \in D_{u}} \gamma_{j i} \boldsymbol{x}_{j}+\sum_{\left(\boldsymbol{x}_{i}, y_{i}\right) \in D_{l} \wedge y_{j}=i} \boldsymbol{x}_{j}\right)
\]\[
\begin{aligned} \boldsymbol{\Sigma}_{i}=\frac{1}{\sum_{\boldsymbol{x}_{j} \in D_{u}} \gamma_{j i}+l_{i}} & \left(\sum_{\boldsymbol{x}_{j} \in D_{u}} \gamma_{j i}\left(\boldsymbol{x}_{j}-\boldsymbol{\mu}_{i}\right)\left(\boldsymbol{x}_{j}-\boldsymbol{\mu}_{i}\right)^{T}\right. \\ & \left.+\sum_{\left(\boldsymbol{x}_{i}, y_{i}\right) \in D_{l} \wedge y_{j}=i}\left(\boldsymbol{x}_{j}-\boldsymbol{\mu}_{i}\right)\left(\boldsymbol{x}_{j}-\boldsymbol{\mu}_{i}\right)^{T}\right)\end{aligned}
\]\[
\alpha_{i}=\frac{1}{m}\left(\sum_{\boldsymbol{x}_{j} \in D_{u}} \gamma_{j i}+l_{i}\right)
\]
}
将高斯混合模型换成混合专家模型, 朴素贝叶斯模型等, 可推导出其他生成式半监督方法。

生成式方法的优点在于简单,易于实现,在有标记数据极少的情形下往往强于其他方法的性能;缺点是要求模型假设必须准确,否则会影响泛化性能。
\section{半监督SVM}\label{sec:12.3}
半监督支持向量机是支持向量机在半监督学习上的推广,试图找到能将两类有标记样本分开且穿过低密度数据区域的划分超平面。

半监督SVM中最著名的是TSVM,是针对二分类问题的学习方法。TSVM对未标记样本进行各种可能的标记指派,即尝试将每个未标记样本分别作为正例与反例,并在所有这些结果中寻求一个在所有样本上间隔最大化的划分超平面。

形式化地来说,给定有标记样本集$D_l = \{(\boldsymbol x_1, y_1),(\boldsymbol x_2, y_2),\ldots,(\boldsymbol x_l, y_l)\}$与无标记数据集$D_u = \{\boldsymbol x_{l+1}, \boldsymbol x_{l+2}, \ldots,\boldsymbol x_{l+u}\}, y\in\{-1, +1\}, l+u=m$,TSVM的学习目标是为$D_u$中的样本给出预测标记$\hat{\boldsymbol y} = (\hat{y}_{l+1},\hat{y}_{l+2},\ldots,\hat{y}_{l+u})$,使得:
\[
\begin{aligned} \min _{\boldsymbol{w}, b, \hat{\boldsymbol{y}}, \boldsymbol{\xi}} & \frac{1}{2}\|\boldsymbol{w}\|_{2}^{2}+C_{l} \sum_{i=1}^{l} \xi_{i}+C_{u} \sum_{i=l+1}^{m} \xi_{i} \\ \text { s.t. } & y_{i}\left(\boldsymbol{w}^{\top} \boldsymbol{x}_{i}+b\right) \geq 1-\xi_{i}, \quad i=1, \ldots, l, \\ & \hat{y}_{i}\left(\boldsymbol{w}^{\top} \boldsymbol{x}_{i}+b\right) \geq 1-\xi_{i}, \quad i=l+1, \ldots, m, \\ & \xi_{i} \geq 0, \quad i=1, \ldots, m,\end{aligned}
\]

TSVM采用局部搜索迭代式地寻找近似解。具体而言:
\begin{itemize}
    \item 先用有标记样本训练一个SVM,为所有无标记样本指派一个伪标记$\hat{y}$并加入数据中进行训练;
    \item 之后设置$C_u \ll C_l$,不断选择标记为异类且容易出错的未标记样本对,交换标记并重新求解超平面$(w,b)$与松弛向量$\xi$,同时逐渐增大$C_u$,直至$C_u = C_l$。
\end{itemize}

半监督SVM可能出现类别不平衡问题,同时搜索标记指派出错的样本对计算开销巨大。
\marginnote{\footnotesize 类别不平衡问题可以将$C_u$拆解为对应正例与负例的$C_u^+, C_u^-$并初始化控制比率来解决。}

\section{图半监督学习}\label{sec:12.4}
将数据集映射为一张图,每个样本对应一个点,若两个样本之间的相似度很高,则对应节点之间存在一条边,边的\textbf{强度}正比于样本之间的\textbf{相似度}。

将所有有标记样本对应的节点想象为“有染色”的节点,半监督学习的过程即对应颜色在图上扩散的过程,由于图与矩阵的对应,我们基于矩阵运算对半监督学习进行推导分析。

\thm{图半监督学习}{thm_ref}{基于有标记数据集与无标记数据集的并$D_l \cup D_u$构建一张图:
\[
G = (V,E), V = \{\boldsymbol x_1,\ldots,\boldsymbol x_l,\boldsymbol x_{l+1},\ldots,\boldsymbol x_{l+u}\}
\]
边集$E$可表示为一个亲和矩阵,基常于高斯函数定义为:\[
\mathbf{W}_{i j}=\left\{\begin{array}{cl}\exp \left(\frac{-\left\|x_{i}-x_{j}\right\|_{2}^{2}}{2 \sigma^{2}}\right), & \text { if } i \neq j \\ 0 & , \text { otherwise }\end{array}\right.
\]
假定从图$G=(V,E)$学的一个实值函数$f:V\to \mathbb{R}$,对应的分类规则为$y_i = \operatorname{sign}(f(\boldsymbol x_i))$,则可定义关于$f$的\textbf{“能量函数”}以最优化学习结果,使得相似的样本拥有相似的标记:
\[
\begin{array}{l}E(f)=\frac{1}{2} \sum_{i=1}^{m} \sum_{j=1}^{m} \mathbf{(W)}_{i j}\left(f\left(\boldsymbol{x}_{i}\right)-f\left(\boldsymbol{x}_{j}\right)\right)^{2} \\ =\frac{1}{2}\left(\sum_{i=1}^{m} d_{i} f^{2}\left(\boldsymbol{x}_{i}\right)+\sum_{j=1}^{m} d_{j} f^{2}\left(\boldsymbol{x}_{j}\right)-2 \sum_{i=1}^{m} \sum_{j=1}^{m} \mathbf{(W)}_{i j} f\left(\boldsymbol{x}_{i}\right) f\left(\boldsymbol{x}_{j}\right)\right) \\ =\boldsymbol{f}^{T}(\mathbf{D}-\mathbf{W}) \boldsymbol{f}\end{array}
\]
其中$\boldsymbol f = (\boldsymbol f_l; \boldsymbol f_u)$分别表示函数$f$在有标记样本与无标记样本上的预测结果,$D$是一个对角矩阵,每一行的对角元素对应矩阵$\mathbf{W}$的第i行元素之和。
将上式转变为分块矩阵的形式,可得:\[
\begin{aligned} E(f) & =\left(f_{l}^{\top} f_{u}^{\top}\right)\left(\left[\begin{array}{cc}\mathbf{D}_{l l} & \mathbf{0}_{l u} \\ \mathbf{0}_{u l} & \mathbf{D}_{u u}\end{array}\right]-\left[\begin{array}{cc}\mathbf{W}_{l l} & \mathbf{W}_{l u} \\ \mathbf{W}_{u l} & \mathbf{W}_{u u}\end{array}\right]\right)\left[\begin{array}{c}f_{l} \\ f_{u}\end{array}\right] \\ & = f_{l}^{\top}\left(\mathbf{D}_{l l}-\mathbf{W}_{l l}\right) f_{l}-2 f_{u}^{\top} \mathbf{W}_{u l} f_{l}+f_{u}^{\top}\left(\mathbf{D}_{u u}-\mathbf{W}_{u u}\right) f_{u} .\end{aligned}
\]
由$ \frac{\partial E(f)}{\partial f_{u}}=\mathbf{0} $可得:
\[ \begin{aligned} \frac{\partial E(f)}{\partial \boldsymbol{f}_{u}} & =\frac{\partial \boldsymbol{f}_{l}^{\mathrm{T}}\left(\boldsymbol{D}_{l l}-\boldsymbol{W}_{l l}\right) \boldsymbol{f}_{l}-2 \boldsymbol{f}_{u}^{\mathrm{T}} \boldsymbol{W}_{u l} \boldsymbol{f}_{l}+\boldsymbol{f}_{u}^{\mathrm{T}}\left(\boldsymbol{D}_{u u}-\boldsymbol{W}_{u u}\right) \boldsymbol{f}_{u}}{\partial \boldsymbol{f}_{u}} \\ & =-2 \boldsymbol{W}_{u l} \boldsymbol{f}_{l}+2\left(\boldsymbol{D}_{u u}-\boldsymbol{W}_{u u}\right) \boldsymbol{f}_{u} = 0 \\
\boldsymbol{f}_{u}&=\left(\mathbf{D}_{u u}-\mathbf{W}_{u u}\right)^{-1} \mathbf{W}_{u l} \boldsymbol{f}_{l}\end{aligned} \]

令\[
\begin{array}{c}\mathbf{P}=\mathbf{D}^{-1} \mathbf{W}=\left[\begin{array}{cc}\mathbf{D}_{l l}^{-1} & 0_{l u} \\ \mathbf{0}_{u l} & \mathbf{D}_{u u}^{-1}\end{array}\right]\left[\begin{array}{cc}\mathbf{W}_{l l} & \mathbf{W}_{l u} \\ \mathbf{W}_{u l} & \mathbf{W}_{u u}\end{array}\right]=\left[\begin{array}{cc}\mathbf{D}_{l l}^{-1} \mathbf{W}_{l l} & \mathbf{D}_{l l}^{-1} \mathbf{W}_{l u} \\ \mathbf{D}_{u u}^{-1} \mathbf{W}_{u l} & \mathbf{D}_{u u}^{-1} \mathbf{W}_{u u}\end{array}\right] \\ \mathbf{P}_{u u}=\mathbf{D}_{u u}^{-1} \mathbf{W}_{u u}, \mathbf{P}_{u l}=\mathbf{D}_{u u}^{-1} \mathbf{W}_{u l}\end{array},
\]带入前式则有\[
\begin{aligned} \boldsymbol{f}_{u} & =\left(\boldsymbol{D}_{u u}\left(\boldsymbol{I}-\boldsymbol{D}_{u u}^{-1} \boldsymbol{W}_{u u}\right)\right)^{-1} \boldsymbol{W}_{u l} \boldsymbol{f}_{l} \\ & =\left(\boldsymbol{I}-\boldsymbol{D}_{u u}^{-1} \boldsymbol{W}_{u u}\right)^{-1} \boldsymbol{D}_{u u}^{-1} \boldsymbol{W}_{u l} \boldsymbol{f}_{l} \\ & =\left(\boldsymbol{I}-\boldsymbol{P}_{u u}\right)^{-1} \boldsymbol{P}_{u l} \boldsymbol{f}_{l}\end{aligned}
\]
}

\marginnote{\footnotesize 上述描述的是二分类问题标记传播方法,还有适用于多分类的迭代式标记传播方法,但这个不会考的}

图半监督学习的优点是:概念清晰,易于使用矩阵分析探索算法性质;

缺点是:图矩阵的存储开销高,建图时仅考虑了训练样本集,而新样本在图中的位置难以估计,可能需要重构。

\section{基于分歧的方法}\label{sec:12.5}

基于分歧的方法使用多学习器,通过学习器之间的“分歧”来利用未标记数据。

这一类方法的重要代表是协同训练,最初是针对“多视图”数据设计的。在现实应用中,一个数据对象往往同时拥有多个属性集,每个属性集构成一个“视图”。假设不同视图拥有\textbf{相容性},即其所包含的关于输出空间$\mathcal{Y}$的信息是一致的,协同训练假设数据拥有两个充分(每个视图都包含足以产生最优学习器的信息)且条件独立(给定类别标记调价下两视图独立)的视图,则可通过将每个视图训练出的学习器最确信的样本交给其他学习器、加入下一轮训练的方式提升泛化性。

理论证明:若两个视图充分且条件独立,则可利用未标记样本通过协同训练将弱分类器的泛化性能提升到任意高。但实际测试发现,不一定需要多视图, 只需要弱学习器之间有显著的分歧,也可通过提供伪标记样本的方式提高泛化性能。

这类方法的缺点是: 有标记样本很少时, 不容易生成有显著分歧、性能尚可的多个学习器。

\section{半监督聚类}\label{sec:12.6}
半监督聚类是在聚类上添加额外监督信息,以获得更好的聚类效果的做法。

聚类任务中获得的监督信息大致有两种类型:

第一种类型是“必连”(must-link)与“勿连”(cannot-link)约束,前者是指样本必属于同一个簇,后者则是指样本必不属于同一个簇;

使用这一类监督信息的代表是约束K均值,它在聚类过程中,在确定样本属于的簇时,首先确保必连与勿连关系集合的约束必须满足,若不冲突才选择最近簇,否则尝试次近。

第二种类型的监督信息则是少量的有标记样本。

使用这一类监督信息的代表是约束种子K均值,直接将有标记的样本作为种子初始化K均值算法的K个聚类中心。

\section{本章往年考试题目}\label{sec:12.7}

\ex{2022年考试原题}{ex_ref}{图半监督模型中$E(f) = \boldsymbol{f}^{T}(\mathbf{D}-\mathbf{W}) \boldsymbol{f}$中$E(f)$的含义是什么,D的表达是?}
\ex{2023/2025年考试原题}{ex_ref}{图半监督能量函数的定义是什么?并推导其闭式解。}

$f$是图半监督学习中基于训练数据集生成的图$G=(V,E)$学得的一个实值二分类函数,$E(f) = \boldsymbol{f}^{T}(\mathbf{D}-\mathbf{W}) \boldsymbol{f}$是对函数$f$学习结果的最优化约束,使得相似的样本拥有相似的标记,称为能量函数。

其中$\boldsymbol f = (\boldsymbol f_l; \boldsymbol f_u)$分别表示函数$f$在有标记样本与无标记样本上的预测结果,$D$是一个对角矩阵,每一行的对角元素对应边集E对应的亲和矩阵$\mathbf{W}$的第i行元素之和。

闭式解推导过程详见\ref{sec:12.4}节。





%\partimg{images/pexels-photo-931018.jpeg}
%\part{度量空间和赋范线性空间}
% \input{1.tex}

% \chapter{Appendix}
\section{拓展链接}
\begin{itemize}
    \item 泛函分析概论:
    
    (\url{https://mp.weixin.qq.com/s/ebhG6fFNry0kqKulTzFxeg})
    \item 
\end{itemize}

\newpage
\section{经典反例}



\newpage
\section{其他}

\begin{figure}[!htbp]
	\centering
		\sidecaption{一些常见的赋范空间及其性质\label{fig:7.1}}{\includegraphics[width=\textwidth]{images/Appendix/一些常见的赋范空间与性质.jpg}}
\end{figure}

\begin{figure}[!htbp]
	\centering
		\sidecaption{一些收敛的定义\label{fig:7.2}}{\includegraphics[width=\textwidth]{images/Appendix/{6CB4C3F6-D97B-4C8B-B9CC-49E00321CF45}.png}}
\end{figure}

%-----------------------------------%
\newgeometry{left=2cm,right=2cm,bottom=2cm,top=3cm}
\printbibliography[heading=bibintoc]


\end{document}
